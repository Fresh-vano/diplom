\newsection
\section{Техническое задание}
\subsection{Основание для разработки}

Основанием для разработки веб-платформа для анализа и визуализации статистических данных киберспортивной игры Counter-Strike 2 является задание на выпускную квалификационную работу приказ ректора ЮЗГУ от <<04>> апреля 2024 года № 1620-с <<Об утверждении тем выпускных квалификационных работ и руководителей выпускных квалификационных работ>>.

\subsection{Назначение разработки}

Функциональное назначение разрабатываемой веб-платформы заключается в предоставлении игрокам и киберспортивным тренерам эффективного инструмента отображения статистических данных из матчей с целью повышения эффективности команд и игроков, а так же отображение и предоставление прогнозов на будущие исходы матчей.

Предполагается, что данной веб-платформой будут пользоваться как киберспортивные тренеры и команды, для анализа прошедших матчей и сравнения различных метрик во времени, так и обычные игроки, для получения статистических данных по киберспортивным матчам и предоставления прогнозов на будущие матчи.

Задачами разработки данной веб-платформы являются:
\begin{itemize}
	\item создание базы данных всех киберспортивных матчей;
	\item создание нейронной сети для прогнозирования матчей;
	\item реализация вывода статистики на веб-платформу;
	\item реализация анализа метрик из .dem записей игр.
\end{itemize}

\subsection{Требования к веб-платформе}
\subsubsection{Требования к данным веб-платформы}

Входными данными для веб-платформы являются:
\begin{itemize}
	\item записи прошедших матчей в формате .dem;
	\item данные с информацией о турнире;
	\item метрики игроков из матчей;
	\item данные отправленные пользователем через API.
\end{itemize}

Выходными данными для веб-платформы являются:
\begin{itemize}
	\item статистика игроков из матча;
	\item прогноз на выбранный матч;
	\item данные выбранного турнира;
	\item данные выбранного матча.
\end{itemize}

На рисунке \ref{fig:-conceptual_classes} представлены концептуальные классы backend части веб-платформы.
\begin{figure}
	\centering
	\includegraphics[width=0.9\linewidth]{"images/Концептуальные классы"}
	\caption[Концептуальные классы backend части веб-платформы]{Концептуальные классы backend части веб-платформы}
	\label{fig:-conceptual_classes}
\end{figure}

\subsubsection{Функциональные требования к веб-платформе}

На основании анализа предметной области в разрабатываемой веб-платформы анализа и визуализации статистических данных киберспортивной игры CS2 должны быть реализованы следующие функции:
\begin{itemize}
	\item просмотр информации игрока;
	\item просмотр рейтинга игрока;
	\item добавление игрока для сравнения;
	\item сравнение статистики игроков;
	\item просмотр информации о турнире;
	\item просмотр информации о матче;
	\item просмотр статистики матча;
	\item отображение прогноза на матч;
	\item просмотр информации команды;
	\item просмотр рейтинга команды;
	\item добавление команды для сравнения;
	\item сравнение статистики команд.
\end{itemize}

На рисунке \ref{fig:-use_case_diagram} представлены функциональные требования к системе в виде диаграммы прецедентов нотации UML.
\begin{figure}
	\centering
	\includegraphics[width=0.9\linewidth]{"images/Диаграмма прецедентов"}
	\caption{Диаграмма прецедентов}
	\label{fig:-use_case_diagram}
\end{figure}

\paragraph{Вариант использования «Просмотр информации игрока»}

Заинтересованные лица и их требования: Пользователи веб-платформы, которые хотят получить детальную информацию о игроке.

Предусловие: Пользователь загружает главную страницу сайта.

Постусловие: Пользователь просматривает необходимую информацию о игроке.

Основной успешный сценарий:
\begin{enumerate}
	\item Пользователь заходит на главную страницу.
	\item Пользователь выбирает раздел с информацией об игроках.
	\item Пользователь ищет интересующего его игрока фильтруя игроков по имени, команде или никнейму.
	\item Система отображает профиль с подробной информацией об игроке.
\end{enumerate}

\paragraph{Вариант использования «Просмотр рейтинга игроков»}

Заинтересованные лица и их требования: Пользователи веб-платформы, которые хотят просмотреть рейтинг лучших игроков на данный момент.

Предусловие: Пользователь загружает главную страницу сайта.

Постусловие: Пользователь просматривает рейтинговую таблицу игроков.

Основной успешный сценарий:
\begin{enumerate}
	\item Пользователь заходит на главную страницу.
	\item Пользователь выбирает раздел с рейтинговой таблицей игроков.
	\item Пользователь просматривает рейтинговую таблицу игроков.
\end{enumerate}

\paragraph{Вариант использования «Добавление игрока для сравнения»}

Заинтересованные лица и их требования: Пользователи веб-платформы, которые хотят добавить игрока для сравнения его статистики с другим игроком.

Предусловие: Пользователь находится на странице с рейтингом игроков.

Постусловие: Пользователь добавляет игрока для будущего сравнения.

Основной успешный сценарий:
\begin{enumerate}
	\item Пользователь наводит курсор на игрока, которого хочет добавить для сравнения.
	\item Пользователь нажимает кнопку "Сравнить".
	\item Игрок добавляется в список для сравнения.
\end{enumerate}

\paragraph{Вариант использования «Сравнение статистики игроков»}

Заинтересованные лица и их требования: Пользователи веб-платформы, желающие сравнить статистические данные разных игроков для анализа.

Предусловие: Пользователь добавил 2 или более игроков для сравнения.

Постусловие: Пользователь получил сравнительный анализ статистики выбранных игроков.

Основной успешный сценарий:
\begin{enumerate}
	\item Пользователь переходит в раздел сравнения статистики игроков.
	\item Система предоставляет сравнительный отчет статистических данных для выбранных игроков.
\end{enumerate}

\paragraph{Вариант использования «Просмотр информации о турнире»}

Заинтересованные лица и их требования: Пользователи веб-платформы, желающие просмотреть информацию определенного турнира.

Предусловие: Пользователь загружает главную страницу сайта.

Постусловие: Пользователь просматривает информацию о выбранном турнире.

Основной успешный сценарий:
\begin{enumerate}
	\item Пользователь заходит на главную страницу.
	\item Пользователь выбирает раздел со списком турниров.
	\item Пользователь фильтрует список всех турниров.
	\item Пользователь выбирает интересующий турнир.
	\item Пользователь просматривает информацию о выбранном турнире.
\end{enumerate}

\paragraph{Вариант использования «Просмотр информации о матче»}

Заинтересованные лица и их требования: Пользователи веб-платформы, желающие просмотреть информацию определенного матча.

Предусловие: Пользователь загружает главную страницу сайта.

Постусловие: Пользователь просматривает информацию о выбранном матче.

Основной успешный сценарий:
\begin{enumerate}
	\item Пользователь заходит на главную страницу.
	\item Пользователь выбирает раздел со всеми матчами.
	\item Пользователь выбирает интересующий матч.
	\item Пользователь просматривает информацию о выбранном матче.
\end{enumerate}

\paragraph{Вариант использования «Просмотр статистики матча»}

Заинтересованные лица и их требования: Пользователи веб-платформы, желающие просмотреть и анализировать статистику выбранного прошедшего матча.

Предусловие: Пользователь находится на странице с прошедшим матчем.

Постусловие: Пользователь просмотрел и проанализировал статистику матча.

Основной успешный сценарий:
\begin{enumerate}
	\item Пользователь на странице матча отображает статистику матча.
	\item Система отображает детальную статистику выбранного матча.
\end{enumerate}

\paragraph{Вариант использования «Отображение прогноза на матч»}

Заинтересованные лица и их требования: Пользователи веб-платформы, желающие просмотреть прогноз нейронной сети на будущий матч.

Предусловие: Пользователь находится на странице с будущим матчем.

Постусловие: Пользователь просмотрел прогноз на матча.

Основной успешный сценарий:
\begin{enumerate}
	\item Пользователь на странице матча отображает прогноз на данный матч.
	\item Система отображает прогноз на выбранный матч.
\end{enumerate}

\paragraph{Вариант использования «Просмотр информации команды»}

Заинтересованные лица и их требования: Пользователи веб-платформы, которые хотят получить детальную информацию о выбранной команде.

Предусловие: Пользователь загружает главную страницу сайта.

Постусловие: Пользователь просматривает необходимую информацию о команде.

Основной успешный сценарий:
\begin{enumerate}
	\item Пользователь заходит на главную страницу.
	\item Пользователь выбирает раздел с информацией о командах.
	\item Пользователь ищет интересующую команду.
	\item Система отображает страницу с информацией о команде.
\end{enumerate}

\paragraph{Вариант использования «Просмотр рейтинга команд»}

Заинтересованные лица и их требования: Пользователи веб-платформы, которые хотят просмотреть рейтинг лучших команд на данный момент.

Предусловие: Пользователь загружает главную страницу сайта.

Постусловие: Пользователь просматривает рейтинговую таблицу команд.

Основной успешный сценарий:
\begin{enumerate}
	\item Пользователь заходит на главную страницу.
	\item Пользователь выбирает раздел с рейтинговой таблицей команд.
	\item Пользователь просматривает рейтинговую таблицу команд.
\end{enumerate}

\paragraph{Вариант использования «Добавление команды для сравнения»}

Заинтересованные лица и их требования: Пользователи веб-платформы, которые хотят добавить команду для сравнения ее статистики среди других команд.

Предусловие: Пользователь находится на странице с рейтингом команд.

Постусловие: Пользователь добавляет команду для будущего сравнения.

Основной успешный сценарий:
\begin{enumerate}
	\item Пользователь наводит курсор на команду, которую хочет добавить для сравнения.
	\item Пользователь нажимает кнопку "Сравнить".
	\item Команда добавляется в список для сравнения.
\end{enumerate}

\paragraph{Вариант использования «Сравнение статистики команд»}

Заинтересованные лица и их требования: Пользователи веб-платформы, желающие сравнить статистику команд для анализа.

Предусловие: Пользователь добавил 2 или более команд для сравнения.

Постусловие: Пользователь получил сравнительный анализ статистики выбранных команд.

Основной успешный сценарий:
\begin{enumerate}
	\item Пользователь переходит в раздел сравнения статистики команд.
	\item Система предоставляет сравнительный отчет статистических данных для выбранных команд.
\end{enumerate}

\subsubsection{Требования пользователя к интерфейсу веб-платформы}

В веб-платформе должны присутствовать следующие графические интерфейсы взаимодействия с пользователем:

\begin{enumerate}
	\item Главная страница с актуальной информацией о текущих киберспортивных турнирах и меню навигации среди доступных функций.
	\item Страница списка турниров с группировкой на прошедшие, текущие и будущие.
	\item Страница турнира с отображением информации, участвующих командах и предстоящих матчах.
	\item Страница матча с отображением информации, команд, прогнозе на данный матч и сводной статистики для данных команд.
	\item Страница с рейтингом лучших команд.
	\item Страница с рейтингом лучших игроков.
	\item Модуль интерактивных диаграмм и графики для детализации статистики игроков и команд.
	\item Страница команды с основной информацией, местом в рейтинге, последними матчами и статистикой команды.
	\item Страница игрока с основной информацией, местом в рейтинге, последними матчами и статистикой игрока.
	\item Страница сравнения метрик между выбранными игроками или командами.
\end{enumerate}

Все страницы должны иметь адаптивную верстку для взаимодействия с платформой с разных устройств. Каждая страница должны иметь основные принципы UI/UX дизайна.
Во время загрузки страницы должен быть реализован каркасный экран загрузки, с анимированным отображением скелета сайта до загрузки всех остальных данных с сервера. Загрузочный экран метрик будет сопровождаться индикатором прогресса обработки данных. Интерфейс результатов анализа должен позволять легко переключаться между различными видами статистики и визуализаций.

\subsection{Нефункциональные требования к программной системе}
\subsubsection{Требования к архитектуре}
Веб-платформа должна быть выполнена в микросервисной архитектуре. Между собой микросервисы должны общаться через gRPC. Каждый микросервис будет упакован в свой Docker-контейнер, что облегчит развертывание, масштабирование и обеспечение изоляции зависимостей. Архитектура должна поддерживать автоматическое масштабирование и самовосстановление, а также обеспечивать высокую доступность и отказоустойчивость.

\subsubsection{Требования к надежности}

При использовании веб-платформы должна быть обеспечена стабильная работа серверов - более 99\% от всего времени работы, регулярное создание резервных копий данных, защита от несанкционированного доступа клиентов к неконтролируемой выгрузке статистики из базы данных в обход API.

Платформа должна обрабатывать и выдерживать высокую нагрузку - до 100 запросов в секунду.

\subsubsection{Требования к программному обеспечению}

Для реализации frontend части веб-платформы должен быть использован язык JavaScript и фреймворк React. 

Для реализации backend части веб-платформы должны использоваться следующие языки программирования: 
\begin{itemize}
	\item C\# - для разработки API и модуля получения новых турниров.
	\item Goland - для разработки анализатора .dem файлов.
	\item Python - для разработки нейронной сети для прогнозов.
\end{itemize}

\subsubsection{Требования к аппаратному обеспечению}

Для открытия веб-платформы потребуется устройство с доступом в сеть интернет, а так же браузер Google Chrome, Mozilla Firefox или Microsoft Edge с поддержкой выполнения JavaScript.

Для работы backend части веб-платформы необходим облачный сервер на операционной системе Linux с установленным Docker для развертывания всех сервисов. Так же дисковое пространство не менее 5 Гб, свободная оперативная память в размере не менее 6 Гб, видеокарта с не менее 1024 Мб видеопамяти.

\subsubsection{Требования к оформлению документации}

Разработка программной документации и программного изделия должна производиться согласно ГОСТ 19.102-77 и ГОСТ 34.601-90. Единая система программной документации.
