\newsection
\section{Рабочий проект}
\subsection{Описание программных классов веб-платформы}
\subsubsection{Описание классов для взаимодействия с базой данных}

В данном разделе представлены основные классы для взаимодействия с базой данных, используемые во всей системе для представления различных данных, связанных с киберспортивными турнирами, матчами, командами и игроками. Эти классы описываются с указанием их атрибутов, типа данных и ключевых свойств.

\begin{xltabular}{\textwidth}{|X|X|X|X|}
	\caption{Свойства класса BaseModel}\label{table:BaseModel}\\ \hline
	\thead{Свойство} & \thead{Тип} & \thead{Обязательное} & \thead{Описание} \\ \hline
	\thead{1} & \thead{2} & \thead{3} & \thead{4} \\ \hline
	\endfirsthead
	\continuecaption{Продолжение таблицы \ref{table:BaseModel}} \hline
	\thead{1} & \thead{2} & \thead{3} & \thead{4} \\ \hline
	\endhead
	CreatedAt & DateTimeOffset & true & Дата и время создания записи \\ \hline
	UpdatedAt & DateTimeOffset & true & Дата и время обновления записи \\ \hline
\end{xltabular}

\begin{xltabular}{\textwidth}{|X|X|X|X|}
	\caption{Свойства класса Country}\label{table:Country}\\ \hline
	\thead{Свойство} & \thead{Тип} & \thead{Обязательное} & \thead{Описание} \\ \hline
	\thead{1} & \thead{2} & \thead{3} & \thead{4} \\ \hline
	\endfirsthead
	\continuecaption{Продолжение таблицы \ref{table:Country}} \hline
	\thead{1} & \thead{2} & \thead{3} & \thead{4} \\ \hline
	\endhead
	Id & int & true & Уникальный идентификатор страны \\ \hline
	CountryName & string & false & Название страны \\ \hline
	Code & string & true & Код страны \\ \hline
\end{xltabular}

\begin{xltabular}{\textwidth}{|X|X|X|X|}
	\caption{Свойства класса TeamName}\label{table:TeamName}\\ \hline
	\thead{Свойство} & \thead{Тип} & \thead{Обязательное} & \thead{Описание} \\ \hline
	\thead{1} & \thead{2} & \thead{3} & \thead{4} \\ \hline
	\endfirsthead
	\continuecaption{Продолжение таблицы \ref{table:TeamName}} \hline
	\thead{1} & \thead{2} & \thead{3} & \thead{4} \\ \hline
	\endhead
	Id & int & true & Уникальный идентификатор альтернативного названия команды \\ \hline
	Name & string & true & Альтернативное название команды \\ \hline
	TeamId & int & true & Идентификатор команды \\ \hline
	Team & Team & false & Связь с основной сущностью команды \\ \hline
\end{xltabular}

\begin{xltabular}{\textwidth}{|X|X|X|X|}
	\caption{Свойства класса Stage}\label{table:Stage}\\ \hline
	\thead{Свойство} & \thead{Тип} & \thead{Обязательное} & \thead{Описание} \\ \hline
	\thead{1} & \thead{2} & \thead{3} & \thead{4} \\ \hline
	\endfirsthead
	\continuecaption{Продолжение таблицы \ref{table:Stage}} \hline
	\thead{1} & \thead{2} & \thead{3} & \thead{4} \\ \hline
	\endhead
	Id & int & true & Уникальный идентификатор этапа \\ \hline
	Status & Tournament
	StatusEnum & true & Статус турнира \\ \hline
	Title & string & true & Название этапа \\ \hline
	TournamentId & int & true & Идентификатор турнира \\ \hline
	Tournament & Tournament & false & Связь с основной сущностью турнира \\ \hline
	Teams & List<Team> & false & Список команд, участвующих в этапе \\ \hline
	StageType & StageTypeEnum & true & Тип этапа \\ \hline
	Matches & List<Match> & false & Список матчей на этапе \\ \hline
	Description & string & false & Описание этапа \\ \hline
\end{xltabular}

\begin{xltabular}{\textwidth}{|X|X|X|X|}
	\caption{Свойства класса Player}\label{table:Player}\\ \hline
	\thead{Свойство} & \thead{Тип} & \thead{Обязательное} & \thead{Описание} \\ \hline
	\thead{1} & \thead{2} & \thead{3} & \thead{4} \\ \hline
	\endfirsthead
	\continuecaption{Продолжение таблицы \ref{table:Player}} \hline
	\thead{1} & \thead{2} & \thead{3} & \thead{4} \\ \hline
	\endhead
	Id & int & true & Уникальный идентификатор игрока \\ \hline
	Nickname & string & true & Никнейм игрока \\ \hline
	FirstName & string & false & Имя игрока \\ \hline
	LastName & string & false & Фамилия игрока \\ \hline
	Slug & string & true & Уникальный слаг для игрока \\ \hline
	ImageData & string & false & Изображение игрока в формате Base64 \\ \hline
	ImageUrl & string & false & URL изображения игрока \\ \hline
	TeamId & int & false & Идентификатор команды \\ \hline
	Team & Team & false & Связь с основной сущностью команды \\ \hline
	CountryId & int & false & Идентификатор страны \\ \hline
	Country & Country & false & Связь с основной сущностью страны \\ \hline
\end{xltabular}

\begin{xltabular}{\textwidth}{|X|X|X|X|}
	\caption{Свойства класса Team}\label{table:Team}\\ \hline
	\thead{Свойство} & \thead{Тип} & \thead{Обязательное} & \thead{Описание} \\ \hline
	\thead{1} & \thead{2} & \thead{3} & \thead{4} \\ \hline
	\endfirsthead
	\continuecaption{Продолжение таблицы \ref{table:Team}} \hline
	\thead{1} & \thead{2} & \thead{3} & \thead{4} \\ \hline
	\endhead
	Id & int & true & Уникальный идентификатор команды \\ \hline
	Slug & string & true & Уникальный слаг для команды \\ \hline
	Name & string & true & Название команды \\ \hline
	Acronym & string & false & Аббревиатура команды \\ \hline
	AlternativeNames & List<TeamName> & false & Список альтернативных названий команды \\ \hline
	CountryId & int & false & Идентификатор страны \\ \hline
	Country & Country & false & Связь с основной сущностью страны \\ \hline
	IconUrl & string & false & URL иконки команды \\ \hline
	Players & List<Player> & false & Список игроков команды \\ \hline
	WebsiteUrl & string & false & URL веб-сайта команды \\ \hline
	YoutubeUrl & string & false & URL YouTube канала команды \\ \hline
	Tournaments & List<Tournament> & false & Список турниров, в которых участвовала команда \\ \hline
\end{xltabular}

\begin{xltabular}{\textwidth}{|X|X|X|X|}
	\caption{Свойства класса Tournament}\label{table:Tournament}\\ \hline
	\thead{Свойство} & \thead{Тип} & \thead{Обязательное} & \thead{Описание} \\ \hline
	\thead{1} & \thead{2} & \thead{3} & \thead{4} \\ \hline
	\endfirsthead
	\continuecaption{Продолжение таблицы \ref{table:Tournament}} \hline
	\thead{1} & \thead{2} & \thead{3} & \thead{4} \\ \hline
	\endhead
	Id & int & true & Уникальный идентификатор турнира \\ \hline
	Name & string & true & Название турнира \\ \hline
	Slug & string & true & Уникальный слаг для турнира \\ \hline
	Status & Tournament
	StatusEnum & true & Статус турнира \\ \hline
	Tier & TierEnum & true & Уровень турнира \\ \hline
	StartDate & DateTimeOffset & true & Дата начала турнира \\ \hline
	EndDate & DateTimeOffset & true & Дата окончания турнира \\ \hline
	Prize & int & true & Призовой фонд турнира \\ \hline
	EventType & EventTypeEnum & false & Тип события \\ \hline
	Description & string & false & Описание турнира \\ \hline
	Discipline & DisciplineEnum & false & Дисциплина турнира \\ \hline
	ImageUrl & string & false & URL изображения турнира \\ \hline
	Country & Country & false & Страна проведения турнира \\ \hline
	Teams & List<Team> & false & Список команд, участвующих в турнире \\ \hline
	Stages & List<Stage> & false & Список этапов турнира \\ \hline
\end{xltabular}

\begin{xltabular}{\textwidth}{|X|X|c|X|}
	\caption{Свойства класса Map}\label{table:Map}\\ \hline
	\thead{Свойство} & \thead{Тип} & \thead{Обязательное} & \thead{Описание} \\ \hline
	\thead{1} & \thead{2} & \thead{3} & \thead{4} \\ \hline
	\endfirsthead
	\continuecaption{Продолжение таблицы \ref{table:Map}} \hline
	\thead{1} & \thead{2} & \thead{3} & \thead{4} \\ \hline
	\endhead
	Id & int & true & Уникальный идентификатор карты \\ \hline
	MatchID & int & true & Идентификатор матча \\ \hline
	Match & Match & false & Связь с основной сущностью матча \\ \hline
	BeginAt & DateTimeOffset & true & Время начала карты \\ \hline
	Status & MapStatusEnum & true & Статус карты \\ \hline
	MapName & MapNameEnum & true & Название карты \\ \hline
	WinnerScore & int & true & Счет победителя \\ \hline
	LoserScore & int & true & Счет проигравшего \\ \hline
	WinnerId & int & true & Идентификатор команды-победителя \\ \hline
	Winner & Team & false & Связь с командой-победителем \\ \hline
	LoserId & int & true & Идентификатор команды-проигравшей \\ \hline
	Loser & Team & false & Связь с командой-проигравшей \\ \hline
	Number & int & true & Номер карты \\ \hline
	RoundCount & int & true & Количество раундов \\ \hline
	Discipline & DisciplineEnum & true & Дисциплина карты \\ \hline
	PlayerMetrics & List<PlayerMetric> & false & Метрики игроков на карте \\ \hline
	PlayerResult
	Metrics & List<Player
	ResultMetric> & false & Результаты игроков на карте \\ \hline
	Rounds & List<Round> & false & Раунды карты \\ \hline
	DemoName & string & false & Название демо-записи карты \\ \hline
\end{xltabular}

\begin{xltabular}{\textwidth}{|X|X|X|X|}
	\caption{Свойства класса Match}\label{table:Match}\\ \hline
	\thead{Свойство} & \thead{Тип} & \thead{Обязательное} & \thead{Описание} \\ \hline
	\thead{1} & \thead{2} & \thead{3} & \thead{4} \\ \hline
	\endfirsthead
	\continuecaption{Продолжение таблицы \ref{table:Match}} \hline
	\thead{1} & \thead{2} & \thead{3} & \thead{4} \\ \hline
	\endhead
	Id & int & true & Уникальный идентификатор матча \\ \hline
	Slug & string & true & Уникальный слаг для матча \\ \hline
	Team1Id & int & true & Идентификатор первой команды \\ \hline
	Team1 & Team & false & Связь с первой командой \\ \hline
	Team2Id & int & true & Идентификатор второй команды \\ \hline
	Team2 & Team & false & Связь со второй командой \\ \hline
	WinnerTeamId & int & false & Идентификатор команды-победителя \\ \hline
	WinnerTeam & Team & false & Связь с командой-победителем \\ \hline
	LoserTeamId & int & false & Идентификатор команды-проигравшей \\ \hline
	LoserTeam & Team & false & Связь с командой-проигравшей \\ \hline
	Team1Score & int & true & Счет первой команды \\ \hline
	Team2Score & int & true & Счет второй команды \\ \hline
	Status & MatchStatusEnum & true & Статус матча \\ \hline
	BOType & int & true & Тип формата "лучший из" (BO1, BO3 и т.д.) \\ \hline
	StartDate & DateTimeOffset & true & Дата начала матча \\ \hline
	EndDate & DateTimeOffset & true & Дата окончания матча \\ \hline
	ParsedStatus & ParsedStatusEnum & true & Статус парсинга матча \\ \hline
	Url & string & false & URL матча \\ \hline
	DemoUrl & string & false & URL демо-записи матча \\ \hline
	Maps & List<Map> & false & Список карт в матче \\ \hline
	Discipline & DisciplineEnum & true & Дисциплина матча \\ \hline
	TournamentID & int & true & Идентификатор турнира \\ \hline
	Tournament & Tournament & false & Связь с основной сущностью турнира \\ \hline
\end{xltabular}

\begin{xltabular}{\textwidth}{|X|X|X|X|}
	\caption{Свойства класса PlayerMetric}\label{table:PlayerMetric}\\ \hline
	\thead{Свойство} & \thead{Тип} & \thead{Обязательное} & \thead{Описание} \\ \hline
	\thead{1} & \thead{2} & \thead{3} & \thead{4} \\ \hline
	\endfirsthead
	\continuecaption{Продолжение таблицы \ref{table:PlayerMetric}} \hline
	\thead{1} & \thead{2} & \thead{3} & \thead{4} \\ \hline
	\endhead
	Id & int & true & Уникальный идентификатор метрики игрока \\ \hline
	MapId & int & true & Идентификатор карты \\ \hline
	PlayerId & int & true & Идентификатор игрока \\ \hline
	Kills & int & true & Количество убийств \\ \hline
	Deaths & int & true & Количество смертей \\ \hline
	Assists & int & true & Количество ассистов \\ \hline
	Headshots & int & true & Количество выстрелов в голову \\ \hline
	Damage & int & true & Нанесенный урон \\ \hline
\end{xltabular}

\begin{xltabular}{\textwidth}{|X|X|X|X|}
	\caption{Свойства класса PlayerResultMetric}\label{table:PlayerResultMetric}\\ \hline
	\thead{Свойство} & \thead{Тип} & \thead{Обязательное} & \thead{Описание} \\ \hline
	\thead{1} & \thead{2} & \thead{3} & \thead{4} \\ \hline
	\endfirsthead
	\continuecaption{Продолжение таблицы \ref{table:PlayerResultMetric}} \hline
	\thead{1} & \thead{2} & \thead{3} & \thead{4} \\ \hline
	\endhead
	Id & int & true & Уникальный идентификатор результата метрики игрока \\ \hline
	MapId & int & true & Идентификатор карты \\ \hline
	PlayerId & int & true & Идентификатор игрока \\ \hline
	RoundsPlayed & int & true & Количество сыгранных раундов \\ \hline
	RoundsWon & int & true & Количество выигранных раундов \\ \hline
	Kills & int & true & Количество убийств \\ \hline
	Deaths & int & true & Количество смертей \\ \hline
	Assists & int & true & Количество ассистов \\ \hline
	Headshots & int & true & Количество выстрелов в голову \\ \hline
\end{xltabular}

\begin{xltabular}{\textwidth}{|X|X|X|X|}
	\caption{Свойства класса PlayerStats}\label{table:PlayerStats}\\ \hline
	\thead{Свойство} & \thead{Тип} & \thead{Обязательное} & \thead{Описание} \\ \hline
	\thead{1} & \thead{2} & \thead{3} & \thead{4} \\ \hline
	\endfirsthead
	\continuecaption{Продолжение таблицы \ref{table:PlayerStats}} \hline
	\thead{1} & \thead{2} & \thead{3} & \thead{4} \\ \hline
	\endhead
	Id & int & true & Уникальный идентификатор статистики игрока \\ \hline
	PlayerId & int & true & Идентификатор игрока \\ \hline
	MatchesPlayed & int & true & Количество сыгранных матчей \\ \hline
	Kills & int & true & Количество убийств \\ \hline
	Deaths & int & true & Количество смертей \\ \hline
	Assists & int & true & Количество ассистов \\ \hline
	Headshots & int & true & Количество выстрелов в голову \\ \hline
	Damage & int & true & Нанесенный урон \\ \hline
	MVPs & int & true & Количество титулов MVP \\ \hline
	Rating & float & true & Рейтинг игрока \\ \hline
\end{xltabular}

\begin{xltabular}{\textwidth}{|X|X|X|X|}
	\caption{Свойства класса Round}\label{table:Round}\\ \hline
	\thead{Свойство} & \thead{Тип} & \thead{Обязательное} & \thead{Описание} \\ \hline
	\thead{1} & \thead{2} & \thead{3} & \thead{4} \\ \hline
	\endfirsthead
	\continuecaption{Продолжение таблицы \ref{table:Round}} \hline
	\thead{1} & \thead{2} & \thead{3} & \thead{4} \\ \hline
	\endhead
	Id & int & true & Уникальный идентификатор раунда \\ \hline
	MapId & int & true & Идентификатор карты \\ \hline
	Number & int & true & Номер раунда \\ \hline
	WinningTeamId & int & true & Идентификатор выигравшей команды \\ \hline
	WinningTeam & Team & false & Связь с выигравшей командой \\ \hline
	LosingTeamId & int & true & Идентификатор проигравшей команды \\ \hline
	LosingTeam & Team & false & Связь с проигравшей командой \\ \hline
\end{xltabular}

\begin{xltabular}{\textwidth}{|X|X|X|X|}
	\caption{Свойства класса RoundPlayerMetric}\label{table:RoundPlayerMetric}\\ \hline
	\thead{Свойство} & \thead{Тип} & \thead{Обязательное} & \thead{Описание} \\ \hline
	\thead{1} & \thead{2} & \thead{3} & \thead{4} \\ \hline
	\endfirsthead
	\continuecaption{Продолжение таблицы \ref{table:RoundPlayerMetric}} \hline
	\thead{1} & \thead{2} & \thead{3} & \thead{4} \\ \hline
	\endhead
	Id & int & true & Уникальный идентификатор метрики игрока за раунд \\ \hline
	RoundId & int & true & Идентификатор раунда \\ \hline
	PlayerId & int & true & Идентификатор игрока \\ \hline
	Kills & int & true & Количество убийств \\ \hline
	Deaths & int & true & Количество смертей \\ \hline
	Assists & int & true & Количество ассистов \\ \hline
	Headshots & int & true & Количество выстрелов в голову \\ \hline
	Damage & int & true & Нанесенный урон \\ \hline
\end{xltabular}

\begin{xltabular}{\textwidth}{|X|X|X|X|}
	\caption{Свойства класса RoundTeamMetric}\label{table:RoundTeamMetric}\\ \hline
	\thead{Свойство} & \thead{Тип} & \thead{Обязательное} & \thead{Описание} \\ \hline
	\thead{1} & \thead{2} & \thead{3} & \thead{4} \\ \hline
	\endfirsthead
	\continuecaption{Продолжение таблицы \ref{table:RoundTeamMetric}} \hline
	\thead{1} & \thead{2} & \thead{3} & \thead{4} \\ \hline
	\endhead
	Id & int & true & Уникальный идентификатор метрики команды за раунд \\ \hline
	RoundId & int & true & Идентификатор раунда \\ \hline
	TeamId & int & true & Идентификатор команды \\ \hline
	Kills & int & true & Количество убийств \\ \hline
	Deaths & int & true & Количество смертей \\ \hline
	Assists & int & true & Количество ассистов \\ \hline
	Headshots & int & true & Количество выстрелов в голову \\ \hline
	Damage & int & true & Нанесенный урон \\ \hline
\end{xltabular}

\begin{xltabular}{\textwidth}{|X|X|X|X|}
	\caption{Свойства класса TeamResultMetric}\label{table:TeamResultMetric}\\ \hline
	\thead{Свойство} & \thead{Тип} & \thead{Обязательное} & \thead{Описание} \\ \hline
	\thead{1} & \thead{2} & \thead{3} & \thead{4} \\ \hline
	\endfirsthead
	\continuecaption{Продолжение таблицы \ref{table:TeamResultMetric}} \hline
	\thead{1} & \thead{2} & \thead{3} & \thead{4} \\ \hline
	\endhead
	Id & int & true & Уникальный идентификатор результата метрики команды \\ \hline
	MapId & int & true & Идентификатор карты \\ \hline
	TeamId & int & true & Идентификатор команды \\ \hline
	RoundsPlayed & int & true & Количество сыгранных раундов \\ \hline
	RoundsWon & int & true & Количество выигранных раундов \\ \hline
	Kills & int & true & Количество убийств \\ \hline
	Deaths & int & true & Количество смертей \\ \hline
	Assists & int & true & Количество ассистов \\ \hline
	Headshots & int & true & Количество выстрелов в голову \\ \hline
\end{xltabular}

\subsubsection{Описание DTO классов сервиса анализа .dem файла}

\begin{xltabular}{\textwidth}{|X|X|X|X|}
	\caption{Свойства класса KillEvent}\label{table:KillEvent}\\ \hline
	\thead{Свойство} & \thead{Тип} & \thead{Обязательное} & \thead{Описание} \\ \hline
	\thead{1} & \thead{2} & \thead{3} & \thead{4} \\ \hline
	\endfirsthead
	\continuecaption{Продолжение таблицы \ref{table:KillEvent}} \hline
	\thead{1} & \thead{2} & \thead{3} & \thead{4} \\ \hline
	\endhead
	Id & int & true & Уникальный идентификатор события убийства \\ \hline
	PlayerId & int & true & Идентификатор игрока \\ \hline
	VictimId & int & true & Идентификатор жертвы \\ \hline
	Weapon & string & true & Оружие, использованное в убийстве \\ \hline
	Headshot & bool & true & Было ли убийство выстрелом в голову \\ \hline
\end{xltabular}

\begin{xltabular}{\textwidth}{|X|X|X|X|}
	\caption{Свойства класса EntryKillEvent}\label{table:EntryKillEvent}\\ \hline
	\thead{Свойство} & \thead{Тип} & \thead{Обязательное} & \thead{Описание} \\ \hline
	\thead{1} & \thead{2} & \thead{3} & \thead{4} \\ \hline
	\endfirsthead
	\continuecaption{Продолжение таблицы \ref{table:EntryKillEvent}} \hline
	\thead{1} & \thead{2} & \thead{3} & \thead{4} \\ \hline
	\endhead
	Id & int & true & Уникальный идентификатор события первого убийства \\ \hline
	PlayerId & int & true & Идентификатор игрока \\ \hline
	VictimId & int & true & Идентификатор жертвы \\ \hline
	Weapon & string & true & Оружие, использованное в убийстве \\ \hline
	Time & float & true & Время события \\ \hline
\end{xltabular}

\begin{xltabular}{\textwidth}{|X|X|X|X|}
	\caption{Свойства класса RoundEndReason}\label{table:RoundEndReason}\\ \hline
	\thead{Свойство} & \thead{Тип} & \thead{Обязательное} & \thead{Описание} \\ \hline
	\thead{1} & \thead{2} & \thead{3} & \thead{4} \\ \hline
	\endfirsthead
	\continuecaption{Продолжение таблицы \ref{table:RoundEndReason}} \hline
	\thead{1} & \thead{2} & \thead{3} & \thead{4} \\ \hline
	\endhead
	Id & int & true & Уникальный идентификатор причины окончания раунда \\ \hline
	Reason & string & true & Описание причины \\ \hline
\end{xltabular}

\begin{xltabular}{\textwidth}{|X|X|X|X|}
	\caption{Свойства класса RoundType}\label{table:RoundType}\\ \hline
	\thead{Свойство} & \thead{Тип} & \thead{Обязательное} & \thead{Описание} \\ \hline
	\thead{1} & \thead{2} & \thead{3} & \thead{4} \\ \hline
	\endfirsthead
	\continuecaption{Продолжение таблицы \ref{table:RoundType}} \hline
	\thead{1} & \thead{2} & \thead{3} & \thead{4} \\ \hline
	\endhead
	Id & int & true & Уникальный идентификатор типа раунда \\ \hline
	Type & string & true & Тип раунда (например, пистолетный, эко-раунд и т.д.) \\ \hline
\end{xltabular}

\begin{xltabular}{\textwidth}{|X|X|X|X|}
	\caption{Свойства класса WeaponFireEvent}\label{table:WeaponFireEvent}\\ \hline
	\thead{Свойство} & \thead{Тип} & \thead{Обязательное} & \thead{Описание} \\ \hline
	\thead{1} & \thead{2} & \thead{3} & \thead{4} \\ \hline
	\endfirsthead
	\continuecaption{Продолжение таблицы \ref{table:WeaponFireEvent}} \hline
	\thead{1} & \thead{2} & \thead{3} & \thead{4} \\ \hline
	\endhead
	Id & int & true & Уникальный идентификатор события выстрела \\ \hline
	PlayerId & int & true & Идентификатор игрока \\ \hline
	Weapon & string & true & Оружие, использованное в выстреле \\ \hline
	Time & float & true & Время события \\ \hline
\end{xltabular}

\begin{xltabular}{\textwidth}{|X|X|X|X|}
	\caption{Свойства структуры ClutchEvent}\label{table:ClutchEvent}\\ \hline
	\thead{Свойство} & \thead{Тип} & \thead{Обязательное} & \thead{Описание} \\ \hline
	\thead{1} & \thead{2} & \thead{3} & \thead{4} \\ \hline
	\endfirsthead
	\continuecaption{Продолжение таблицы \ref{table:ClutchEvent}} \hline
	\thead{1} & \thead{2} & \thead{3} & \thead{4} \\ \hline
	\endhead
	Id & int & true & Идентификатор события \\ \hline
	PlayerId & int & true & Идентификатор игрока \\ \hline
	RoundId & int & true & Идентификатор раунда \\ \hline
	EventTime & time.Time & true & Время события \\ \hline
	Outcome & string & true & Исход события \\ \hline
\end{xltabular}

\begin{xltabular}{\textwidth}{|X|X|X|X|}
	\caption{Свойства структуры PlayerHurtedEvent}\label{table:PlayerHurtedEvent}\\ \hline
	\thead{Свойство} & \thead{Тип} & \thead{Обязательное} & \thead{Описание} \\ \hline
	\thead{1} & \thead{2} & \thead{3} & \thead{4} \\ \hline
	\endfirsthead
	\continuecaption{Продолжение таблицы \ref{table:PlayerHurtedEvent}} \hline
	\thead{1} & \thead{2} & \thead{3} & \thead{4} \\ \hline
	\endhead
	Id & int & true & Идентификатор события \\ \hline
	PlayerId & int & true & Идентификатор игрока \\ \hline
	RoundId & int & true & Идентификатор раунда \\ \hline
	EventTime & time.Time & true & Время события \\ \hline
	Damage & int & true & Нанесенный урон \\ \hline
	AttackerId & int & true & Идентификатор атакующего игрока \\ \hline
\end{xltabular}

\subsubsection{Описание DTO классов сервиса поиска информации о турнирах}

\begin{xltabular}{\textwidth}{|X|X|X|X|}
	\caption{Свойства класса StageDto}\label{table:StageDto}\\ \hline
	\thead{Свойство} & \thead{Тип} & \thead{Обязательное} & \thead{Описание} \\ \hline
	\thead{1} & \thead{2} & \thead{3} & \thead{4} \\ \hline
	\endfirsthead
	\continuecaption{Продолжение таблицы \ref{table:StageDto}} \hline
	\thead{1} & \thead{2} & \thead{3} & \thead{4} \\ \hline
	\endhead
	Id & int & true & Уникальный идентификатор этапа \\ \hline
	Title & string & true & Название этапа \\ \hline
	FormatType & string & true & Тип формата этапа \\ \hline
	Status & string & true & Статус этапа \\ \hline
	Teams & List<TeamDto> & false & Список команд на этапе \\ \hline
\end{xltabular}

\begin{xltabular}{\textwidth}{|X|X|X|X|}
	\caption{Свойства класса TeamClanDto}\label{table:TeamClanDto}\\ \hline
	\thead{Свойство} & \thead{Тип} & \thead{Обязательное} & \thead{Описание} \\ \hline
	\thead{1} & \thead{2} & \thead{3} & \thead{4} \\ \hline
	\endfirsthead
	\continuecaption{Продолжение таблицы \ref{table:TeamClanDto}} \hline
	\thead{1} & \thead{2} & \thead{3} & \thead{4} \\ \hline
	\endhead
	ClanName & string & true & Название клана \\ \hline
	TeamId & int & true & Идентификатор команды \\ \hline
	Id & int & true & Уникальный идентификатор \\ \hline
\end{xltabular}

\begin{xltabular}{\textwidth}{|X|X|X|X|}
	\caption{Свойства класса TeamDto}\label{table:TeamDto}\\ \hline
	\thead{Свойство} & \thead{Тип} & \thead{Обязательное} & \thead{Описание} \\ \hline
	\thead{1} & \thead{2} & \thead{3} & \thead{4} \\ \hline
	\endfirsthead
	\continuecaption{Продолжение таблицы \ref{table:TeamDto}} \hline
	\thead{1} & \thead{2} & \thead{3} & \thead{4} \\ \hline
	\endhead
	Id & int & true & Уникальный идентификатор команды \\ \hline
	Slug & string & true & Уникальный слаг команды \\ \hline
	Name & string & true & Название команды \\ \hline
	IconUrl & string & false & URL иконки команды \\ \hline
	YoutubeUrl & string & false & URL YouTube канала команды \\ \hline
	WebsiteUrl & string & false & URL веб-сайта команды \\ \hline
	DisciplineId & int & true & Идентификатор дисциплины \\ \hline
	Acronym & string & false & Аббревиатура команды \\ \hline
	Country & CountryDto & false & Страна команды \\ \hline
\end{xltabular}

\begin{xltabular}{\textwidth}{|X|X|X|X|}
	\caption{Свойства класса CountryDto}\label{table:CountryDto}\\ \hline
	\thead{Свойство} & \thead{Тип} & \thead{Обязательное} & \thead{Описание} \\ \hline
	\thead{1} & \thead{2} & \thead{3} & \thead{4} \\ \hline
	\endfirsthead
	\continuecaption{Продолжение таблицы \ref{table:CountryDto}} \hline
	\thead{1} & \thead{2} & \thead{3} & \thead{4} \\ \hline
	\endhead
	Id & int & true & Уникальный идентификатор страны \\ \hline
	Code & string & true & Код страны \\ \hline
	Name & string & true & Название страны \\ \hline
\end{xltabular}

\begin{xltabular}{\textwidth}{|X|X|X|X|}
	\caption{Свойства класса MatchFullDto}\label{table:MatchFullDto}\\ \hline
	\thead{Свойство} & \thead{Тип} & \thead{Обязательное} & \thead{Описание} \\ \hline
	\thead{1} & \thead{2} & \thead{3} & \thead{4} \\ \hline
	\endfirsthead
	\continuecaption{Продолжение таблицы \ref{table:MatchFullDto}} \hline
	\thead{1} & \thead{2} & \thead{3} & \thead{4} \\ \hline
	\endhead
	Id & int & true & Уникальный идентификатор матча \\ \hline
	WinnerTeamId & int & true & Идентификатор команды-победителя \\ \hline
	LoserTeamId & int & true & Идентификатор команды-проигравшей \\ \hline
	TournamentID & int & true & Идентификатор турнира \\ \hline
\end{xltabular}

\begin{xltabular}{\textwidth}{|X|X|c|X|}
	\caption{Свойства класса MatchGeneralDto}\label{table:MatchGeneralDto}\\ \hline
	\thead{Свойство} & \thead{Тип} & \thead{Обязательное} & \thead{Описание} \\ \hline
	\thead{1} & \thead{2} & \thead{3} & \thead{4} \\ \hline
	\endfirsthead
	\continuecaption{Продолжение таблицы \ref{table:MatchGeneralDto}} \hline
	\thead{1} & \thead{2} & \thead{3} & \thead{4} \\ \hline
	\endhead
	Total & TotalDto & true & Общая информация \\ \hline
	MatchtMin & List<Match
	MinDto> & true & Список минимальных данных о матчах \\ \hline
\end{xltabular}

\begin{xltabular}{\textwidth}{|X|X|X|X|}
	\caption{Свойства класса PlayerDto}\label{table:PlayerDto}\\ \hline
	\thead{Свойство} & \thead{Тип} & \thead{Обязательное} & \thead{Описание} \\ \hline
	\thead{1} & \thead{2} & \thead{3} & \thead{4} \\ \hline
	\endfirsthead
	\continuecaption{Продолжение таблицы \ref{table:PlayerDto}} \hline
	\thead{1} & \thead{2} & \thead{3} & \thead{4} \\ \hline
	\endhead
	Id & int & true & Идентификатор игрока \\ \hline
	Slug & string & true & Короткое имя игрока \\ \hline
	Nickname & string & true & Никнейм игрока \\ \hline
	FirstName & string? & false & Имя игрока \\ \hline
	LastName & string? & false & Фамилия игрока \\ \hline
	ImageUrl & string? & false & URL изображения игрока \\ \hline
	TeamId & int? & false & Идентификатор команды \\ \hline
	ImageData & string? & false & Данные изображения игрока \\ \hline
	Country & CountryDto & true & Страна игрока \\ \hline
\end{xltabular}

\begin{xltabular}{\textwidth}{|X|X|X|X|}
	\caption{Свойства класса MapDto}\label{table:MapDto}\\ \hline
	\thead{Свойство} & \thead{Тип} & \thead{Обязательное} & \thead{Описание} \\ \hline
	\thead{1} & \thead{2} & \thead{3} & \thead{4} \\ \hline
	\endfirsthead
	\continuecaption{Продолжение таблицы \ref{table:MapDto}} \hline
	\thead{1} & \thead{2} & \thead{3} & \thead{4} \\ \hline
	\endhead
	Id & int & true & Идентификатор карты \\ \hline
	MatchID & int & true & Идентификатор матча \\ \hline
	BeginAt & DateTimeOffset & true & Дата и время начала \\ \hline
	Status & string & true & Статус карты \\ \hline
	MapName & string & true & Название карты \\ \hline
	WinnerScore & int & true & Счет победившей команды \\ \hline
	LoserScore & int & true & Счет проигравшей команды \\ \hline
	WinnerTeam & NestedTeamDto & true & Победившая команда \\ \hline
	LoserTeam & NestedTeamDto & true & Проигравшая команда \\ \hline
	Number & int & true & Номер карты \\ \hline
	RoundCount & int & true & Количество раундов \\ \hline
\end{xltabular}

\begin{xltabular}{\textwidth}{|X|X|c|X|}
	\caption{Свойства класса TeamGeneralDto}\label{table:TeamGeneralDto}\\ \hline
	\thead{Свойство} & \thead{Тип} & \thead{Обязательное} & \thead{Описание} \\ \hline
	\thead{1} & \thead{2} & \thead{3} & \thead{4} \\ \hline
	\endfirsthead
	\continuecaption{Продолжение таблицы \ref{table:TeamGeneralDto}} \hline
	\thead{1} & \thead{2} & \thead{3} & \thead{4} \\ \hline
	\endhead
	Total & TotalDto & true & Общая информация \\ \hline
	TeamRanks & List<Team
	RankDto> & true & Список рангов команды \\ \hline
\end{xltabular}

\begin{xltabular}{\textwidth}{|X|X|X|X|}
	\caption{Свойства класса TeamRosterDto}\label{table:TeamRosterDto}\\ \hline
	\thead{Свойство} & \thead{Тип} & \thead{Обязательное} & \thead{Описание} \\ \hline
	\thead{1} & \thead{2} & \thead{3} & \thead{4} \\ \hline
	\endfirsthead
	\continuecaption{Продолжение таблицы \ref{table:TeamRosterDto}} \hline
	\thead{1} & \thead{2} & \thead{3} & \thead{4} \\ \hline
	\endhead
	Players & List<PlayerDto> & true & Список игроков \\ \hline
\end{xltabular}

\begin{xltabular}{\textwidth}{|X|X|X|X|}
	\caption{Свойства класса TotalDto}\label{table:TotalDto}\\ \hline
	\thead{Свойство} & \thead{Тип} & \thead{Обязательное} & \thead{Описание} \\ \hline
	\thead{1} & \thead{2} & \thead{3} & \thead{4} \\ \hline
	\endfirsthead
	\continuecaption{Продолжение таблицы \ref{table:TotalDto}} \hline
	\thead{1} & \thead{2} & \thead{3} & \thead{4} \\ \hline
	\endhead
	Count & int & true & Количество \\ \hline
	Pages & int & true & Количество страниц \\ \hline
	Offset & int & true & Смещение \\ \hline
	Limit & int & true & Лимит \\ \hline
\end{xltabular}

\begin{xltabular}{\textwidth}{|X|X|X|X|}
	\caption{Свойства класса TournamentFullDto}\label{table:TournamentFullDto}\\ \hline
	\thead{Свойство} & \thead{Тип} & \thead{Обязательное} & \thead{Описание} \\ \hline
	\thead{1} & \thead{2} & \thead{3} & \thead{4} \\ \hline
	\endfirsthead
	\continuecaption{Продолжение таблицы \ref{table:TournamentFullDto}} \hline
	\thead{1} & \thead{2} & \thead{3} & \thead{4} \\ \hline
	\endhead
	Id & int & true & Идентификатор турнира \\ \hline
	EventType & string & true & Тип события \\ \hline
	Description & string & true & Описание турнира \\ \hline
	Discipline & string & true & Дисциплина \\ \hline
	ImageUrl & string & true & URL изображения \\ \hline
	Country & int? & false & Идентификатор страны \\ \hline
	Stages & List<StageDto> & true & Список стадий турнира \\ \hline
\end{xltabular}

\begin{xltabular}{\textwidth}{|X|X|X|X|}
	\caption{Свойства класса TournamentGeneralDto}\label{table:TournamentGeneralDto}\\ \hline
	\thead{Свойство} & \thead{Тип} & \thead{Обязательное} & \thead{Описание} \\ \hline
	\thead{1} & \thead{2} & \thead{3} & \thead{4} \\ \hline
	\endfirsthead
	\continuecaption{Продолжение таблицы \ref{table:TournamentGeneralDto}} \hline
	\thead{1} & \thead{2} & \thead{3} & \thead{4} \\ \hline
	\endhead
	Total & TotalDto & true & Общая информация \\ \hline
	TournamentMin & List<Tournament
	MinDto> & true & Список минимальных данных о турнирах \\ \hline
\end{xltabular}

\begin{xltabular}{\textwidth}{|X|X|X|X|}
	\caption{Свойства класса TournamentMinDto}\label{table:TournamentMinDto}\\ \hline
	\thead{Свойство} & \thead{Тип} & \thead{Обязательное} & \thead{Описание} \\ \hline
	\thead{1} & \thead{2} & \thead{3} & \thead{4} \\ \hline
	\endfirsthead
	\continuecaption{Продолжение таблицы \ref{table:TournamentMinDto}} \hline
	\thead{1} & \thead{2} & \thead{3} & \thead{4} \\ \hline
	\endhead
	Id & int & true & Идентификатор турнира \\ \hline
	Name & string & true & Название турнира \\ \hline
	Slug & string & true & Короткое имя турнира \\ \hline
	Status & string & true & Статус турнира \\ \hline
	Tier & string & true & Уровень турнира \\ \hline
	StartDate & DateTimeOffset & true & Дата начала \\ \hline
	EndDate & DateTimeOffset & true & Дата окончания \\ \hline
	Prize & int & true & Призовой фонд \\ \hline
\end{xltabular}

\subsubsection{Описание DTO классов сервиса поиска}

\begin{xltabular}{\textwidth}{|X|X|X|X|}
	\caption{Свойства класса PlayerSearchDto}\label{table:PlayerSearchDto}\\ \hline
	\thead{Свойство} & \thead{Тип} & \thead{Обязательное} & \thead{Описание} \\ \hline
	\thead{1} & \thead{2} & \thead{3} & \thead{4} \\ \hline
	\endfirsthead
	\continuecaption{Продолжение таблицы \ref{table:PlayerSearchDto}} \hline
	\thead{1} & \thead{2} & \thead{3} & \thead{4} \\ \hline
	\endhead
	Id & int & true & Идентификатор игрока \\ \hline
	Nickname & string & true & Никнейм игрока \\ \hline
	FirstName & string? & false & Имя игрока \\ \hline
	LastName & string? & false & Фамилия игрока \\ \hline
	ImageUrl & string? & false & URL изображения игрока \\ \hline
\end{xltabular}

\begin{xltabular}{\textwidth}{|X|X|X|X|}
	\caption{Свойства класса TeamSearchDto}\label{table:TeamSearchDto}\\ \hline
	\thead{Свойство} & \thead{Тип} & \thead{Обязательное} & \thead{Описание} \\ \hline
	\thead{1} & \thead{2} & \thead{3} & \thead{4} \\ \hline
	\endfirsthead
	\continuecaption{Продолжение таблицы \ref{table:TeamSearchDto}} \hline
	\thead{1} & \thead{2} & \thead{3} & \thead{4} \\ \hline
	\endhead
	Id & int & true & Идентификатор команды \\ \hline
	Slug & string & true & Короткое имя команды \\ \hline
	Name & string & true & Название команды \\ \hline
	IconUrl & string? & false & URL иконки команды \\ \hline
\end{xltabular}

\begin{xltabular}{\textwidth}{|X|X|X|X|}
	\caption{Свойства класса TournamentSearchDto}\label{table:TournamentSearchDto}\\ \hline
	\thead{Свойство} & \thead{Тип} & \thead{Обязательное} & \thead{Описание} \\ \hline
	\thead{1} & \thead{2} & \thead{3} & \thead{4} \\ \hline
	\endfirsthead
	\continuecaption{Продолжение таблицы \ref{table:TournamentSearchDto}} \hline
	\thead{1} & \thead{2} & \thead{3} & \thead{4} \\ \hline
	\endhead
	Id & int & true & Идентификатор турнира \\ \hline
	Name & string & true & Название турнира \\ \hline
	Slug & string & true & Короткое имя турнира \\ \hline
	ImageUrl & string? & false & URL изображения турнира \\ \hline
	Prize & int & true & Призовой фонд \\ \hline
	StartDate & DateTimeOffset & true & Дата начала турнира \\ \hline
	EndDate & DateTimeOffset & true & Дата окончания турнира \\ \hline
\end{xltabular}

\begin{xltabular}{\textwidth}{|X|X|X|X|}
	\caption{Свойства класса SearchResult}\label{table:SearchResult}\\ \hline
	\thead{Свойство} & \thead{Тип} & \thead{Обязательное} & \thead{Описание} \\ \hline
	\thead{1} & \thead{2} & \thead{3} & \thead{4} \\ \hline
	\endfirsthead
	\continuecaption{Продолжение таблицы \ref{table:SearchResult}} \hline
	\thead{1} & \thead{2} & \thead{3} & \thead{4} \\ \hline
	\endhead
	Players & List<PlayerSearch
	Dto> & true & Список игроков \\ \hline
	Teams & List<TeamSearch
	Dto> & true & Список команд \\ \hline
	Tournaments & List<Tournament
	SearchDto> & true & Список турниров \\ \hline
\end{xltabular}

\subsection{Спецификация классов веб-платформы}
\subsubsection{Спецификация классов сервиса анализа .dem файла}

Класс \texttt{Demo} предназначен для анализа демо-файлов киберспортивной игры Counter-Strike 2. Он содержит данные о игроках, раундах, командах и различных событиях, происходящих в матче. Класс также включает переменные для расчета рейтингов игроков на основе различных статистик.

\begin{xltabular}{\textwidth}{|X|c|X|X|}
	\caption{Спецификация полей класса Demo}\label{table:DemoFields}\\ \hline
	\thead{Наименование} & \thead{Область \\ видимости} & \thead{Тип данных} & \thead{Описание} \\ \hline
	\thead{1} & \thead{2} & \thead{3} & \thead{4} \\ \hline
	\endfirsthead
	\continuecaption{Продолжение таблицы \ref{table:DemoFields}} \hline
	\thead{1} & \thead{2} & \thead{3} & \thead{4} \\ \hline
	\endhead
	Parser & package & dem.Parser & Парсер демо-файла \\ \hline
	Players & package & []*Models.Player & Список игроков \\ \hline
	Rounds & package & []Models.Round & Список раундов \\ \hline
	Team1 & package & *Models.Team & Первая команда \\ \hline
	Team2 & package & *Models.Team & Вторая команда \\ \hline
	CurrentRound & package & Models.Round & Текущий раунд \\ \hline
	IsFirstKill
	Occurred & package & bool & Произошло первое убийство \\ \hline
	IsFirstShot
	Occured & package & bool & Произошел первый выстрел \\ \hline
	PlayerInClutch1 & package & *Models.Player & Игрок в первом клатче \\ \hline
	PlayerInClutch2 & package & *Models.Player & Игрок во втором клатче \\ \hline
	CurrentClutch & package & *Models.
	ClutchEvent & Текущее событие клатча \\ \hline
	IsMatchStarted & package & bool & Матч начался \\ \hline
	IsFreezetime & package & bool & Время разминки \\ \hline
	KillsThisRound & package & map[*common.
	Player]int & Убийства в этом раунде \\ \hline
	TimeStartKill & package & map[*events.
	PlayerHurt]
	float64 & Время начала убийства \\ \hline
	Kills & package & []Models
	.KillEvent & События убийств \\ \hline
	PlayersHurted & package & []*Models.Player
	HurtedEvent & Раненые игроки \\ \hline
	WeaponFired & package & []Models.Weapon
	FireEvent & События стрельбы из оружия \\ \hline
	Date & package & time.Time & Дата \\ \hline
\end{xltabular}

\begin{xltabular}{\textwidth}{|X|c|X|}
	\caption{Спецификация методов класса Demo}\label{table:DemoMethods}\\ \hline
	\thead{Наименование} & \thead{Метод доступа} & \thead{Описание} \\ \hline
	\thead{1} & \thead{2} & \thead{3} \\ \hline
	\endfirsthead
	\continuecaption{Продолжение таблицы \ref{table:DemoMethods}} \hline
	\thead{1} & \thead{2} & \thead{3} \\ \hline
	\endhead
	ProcessPlayersRating & public & Обработка рейтингов игроков \\ \hline
	ComputeHltvOrgRating & public & Вычисление рейтинга HLTV \\ \hline
	ComputeHltv2OrgRating & public & Вычисление рейтинга 2 HLTV \\ \hline
	ComputeEseaRws & public & Вычисление рейтинга ESEA RWS \\ \hline
	findPlayerBySteamID & private & Поиск игрока по Steam ID \\ \hline
	ProcessTradeKill & public & Обработка убийств на обмене \\ \hline
	ProcessOpenAnd
	EntryKills & public & Обработка первых убийств \\ \hline
	CheckForSpecial
	ClutchEnd & public & Проверка окончания клэтча \\ \hline
	UpdateKillsCount & public & Обновление количества убийств \\ \hline
	HandlePlayerHurted & public & Обработка ранений игрока \\ \hline
	ProcessClutches & public & Обработка клэтчей \\ \hline
	FindAlivePlayerInTeam & public & Поиск живого игрока в команде \\ \hline
	HandlePlayerKilled & public & Обработка убийства игрока \\ \hline
	HandleRoundEnd & public & Обработка окончания раунда \\ \hline
	HandleWeaponFired & public & Обработка стрельбы из оружия \\ \hline
	CalcEquipValue & public & Расчет стоимости экипировки \\ \hline
	GetSideWithHigher
	EquipValue & public & Получение стороны с большей стоимостью экипировки \\ \hline
	GetTeamBySide & public & Получение команды по стороне \\ \hline
\end{xltabular}

Класс TeamsSheet предназначен для подсчета статистики команд в демо-файлах киберспортивной игры Counter-Strike 2 для их сохранения в базу данных.

\begin{xltabular}{\textwidth}{|X|c|X|X|}
	\caption{Спецификация полей класса TeamsSheet}\label{table:TeamsSheetFields}\\ \hline
	\thead{Наименование} & \thead{Область \\ видимости} & \thead{Тип данных} & \thead{Описание} \\ \hline
	\thead{1} & \thead{2} & \thead{3} & \thead{4} \\ \hline
	\endfirsthead
	\continuecaption{Продолжение таблицы \ref{table:TeamsSheetFields}} \hline
	\thead{1} & \thead{2} & \thead{3} & \thead{4} \\ \hline
	\endhead
	Demo & package & *Analyzer.Demo & Демо-объект \\ \hline
	rowPerTeamName & package & map[string]
	*Models
	.TeamSheetRow & Строки статистики для каждой команды \\ \hline
	MapID & package & int & Идентификатор карты \\ \hline
	TeamId & package & int & Идентификатор команды \\ \hline
\end{xltabular}

\begin{xltabular}{\textwidth}{|X|X|X|}
	\caption{Спецификация методов класса TeamsSheet}\label{table:TeamsSheetMethods}\\ \hline
	\thead{Наименование} & \thead{Метод доступа} & \thead{Описание} \\ \hline
	\thead{1} & \thead{2} & \thead{3} \\ \hline
	\endfirsthead
	\continuecaption{Продолжение таблицы \ref{table:TeamsSheetMethods}} \hline
	\thead{1} & \thead{2} & \thead{3} \\ \hline
	\endhead
	NewTeamsSheet & public & Создание нового объекта TeamsSheet \\ \hline
	AddDemo & public & Добавление демо и обновление статистики команды \\ \hline
	UpdateTeamStats & public & Обновление статистики команды \\ \hline
\end{xltabular}

\subsubsection{Спецификация классов сервиса поиска информации о турнирах}

Класс BackgroundServiceController предназначен для управления запуском и остановкой фоновых сервисов обновления данных о командах, турнирах и матчах.

\begin{xltabular}{\textwidth}{|X|X|X|X|}
	\caption{Спецификация полей класса BackgroundServiceController}\label{table:BackgroundServiceControllerFields}\\ \hline
	\thead{Наименование} & \thead{Область \\ видимости} & \thead{Тип данных} & \thead{Описание} \\ \hline
	\thead{1} & \thead{2} & \thead{3} & \thead{4} \\ \hline
	\endfirsthead
	\continuecaption{Продолжение таблицы \ref{table:BackgroundServiceControllerFields}} \hline
	\thead{1} & \thead{2} & \thead{3} & \thead{4} \\ \hline
	\endhead
	\_taskQueue & private & IBackgroundTask
	Queue & Очередь фоновых задач \\ \hline
	\_serviceScope
	Factory & private & IServiceScope
	Factory & Фабрика областей служб \\ \hline
	\_tokens & private & IDictionary<string, CancellationToken
	Source> & Токены отмены для сервисов \\ \hline
\end{xltabular}

\begin{xltabular}{\textwidth}{|X|X|X|}
	\caption{Спецификация методов класса BackgroundServiceController}\label{table:BackgroundServiceControllerMethods}\\ \hline
	\thead{Наименование} & \thead{Метод доступа} & \thead{Описание} \\ \hline
	\thead{1} & \thead{2} & \thead{3} \\ \hline
	\endfirsthead
	\continuecaption{Продолжение таблицы \ref{table:BackgroundServiceControllerMethods}} \hline
	\thead{1} & \thead{2} & \thead{3} \\ \hline
	\endhead
	StartTeamUpdateService & public & Запускает фоновый сервис обновления команд \\ \hline
	StartTournamentUpdate
	Service & public & Запускает фоновый сервис обновления турниров \\ \hline
	StartMatchUpdateService & public & Запускает фоновый сервис обновления матчей \\ \hline
	StopService & public & Останавливает указанный фоновый сервис \\ \hline
	StartService & private & Запускает указанный фоновый сервис с предоставленной функцией работы \\ \hline
\end{xltabular}

Класс HttpClientHelper предназначен для конфигурации HTTP-клиента с необходимыми заголовками.

\begin{xltabular}{\textwidth}{|X|X|X|}
	\caption{Спецификация методов класса HttpClientHelper}\label{table:HttpClientHelperMethods}\\ \hline
	\thead{Наименование} & \thead{Метод доступа} & \thead{Описание} \\ \hline
	\thead{1} & \thead{2} & \thead{3} \\ \hline
	\endfirsthead
	\continuecaption{Продолжение таблицы \ref{table:HttpClientHelperMethods}} \hline
	\thead{1} & \thead{2} & \thead{3} \\ \hline
	\endhead
	ConfigureClient & public & Конфигурирует HTTP-клиент с необходимыми заголовками \\ \hline
\end{xltabular}

Класс MatchUpdateService предназначен для обновления данных о матчах через API.

\begin{xltabular}{\textwidth}{|X|X|X|X|}
	\caption{Спецификация полей класса MatchUpdateService}\label{table:MatchUpdateServiceFields}\\ \hline
	\thead{Наименование} & \thead{Область \\ видимости} & \thead{Тип данных} & \thead{Описание} \\ \hline
	\thead{1} & \thead{2} & \thead{3} & \thead{4} \\ \hline
	\endfirsthead
	\continuecaption{Продолжение таблицы \ref{table:MatchUpdateServiceFields}} \hline
	\thead{1} & \thead{2} & \thead{3} & \thead{4} \\ \hline
	\endhead
	client & private & static HttpClient & HTTP-клиент для запросов \\ \hline
	offset & private & int & Смещение для пагинации \\ \hline
	limit & private & int & Лимит для пагинации \\ \hline
	\_mapper & private & IMapper & Маппер для преобразования DTO \\ \hline
	\_dbContext & private & IApplicationDb
	Context & Контекст базы данных \\ \hline
	count & private & int & Количество матчей для обработки \\ \hline
	\_playerStats & private & PlayerStatsService & Сервис для обработки статистики игроков \\ \hline
\end{xltabular}

\begin{xltabular}{\textwidth}{|X|X|X|}
	\caption{Спецификация методов класса MatchUpdateService}\label{table:MatchUpdateServiceMethods}\\ \hline
	\thead{Наименование} & \thead{Метод доступа} & \thead{Описание} \\ \hline
	\thead{1} & \thead{2} & \thead{3} \\ \hline
	\endfirsthead
	\continuecaption{Продолжение таблицы \ref{table:MatchUpdateServiceMethods}} \hline
	\thead{1} & \thead{2} & \thead{3} \\ \hline
	\endhead
	UpdateMatchAsync & public & Асинхронное обновление данных о матчах \\ \hline
	GetMaps & public & Получение карт для матча \\ \hline
	UpdateMatchMin
	FinishedAsync & public & Асинхронное обновление данных о завершенных матчах \\ \hline
\end{xltabular}

Класс PlayerStatsService предназначен для обновления статистики игроков через API.

\begin{xltabular}{\textwidth}{|X|X|X|X|}
	\caption{Спецификация полей класса PlayerStatsService}\label{table:PlayerStatsServiceFields}\\ \hline
	\thead{Наименование} & \thead{Область \\ видимости} & \thead{Тип данных} & \thead{Описание} \\ \hline
	\thead{1} & \thead{2} & \thead{3} & \thead{4} \\ \hline
	\endfirsthead
	\continuecaption{Продолжение таблицы \ref{table:PlayerStatsServiceFields}} \hline
	\thead{1} & \thead{2} & \thead{3} & \thead{4} \\ \hline
	\endhead
	client & private & static HttpClient & HTTP-клиент для запросов \\ \hline
	\_mapper & private & IMapper & Маппер для преобразования DTO \\ \hline
	\_dbContext & private & IApplicationDb
	Context & Контекст базы данных \\ \hline
\end{xltabular}

\begin{xltabular}{\textwidth}{|X|X|X|}
	\caption{Спецификация методов класса PlayerStatsService}\label{table:PlayerStatsServiceMethods}\\ \hline
	\thead{Наименование} & \thead{Метод доступа} & \thead{Описание} \\ \hline
	\thead{1} & \thead{2} & \thead{3} \\ \hline
	\endfirsthead
	\continuecaption{Продолжение таблицы \ref{table:PlayerStatsServiceMethods}} \hline
	\thead{1} & \thead{2} & \thead{3} \\ \hline
	\endhead
	AddPlayerStatsAsync & public & Асинхронное добавление статистики игрока \\ \hline
\end{xltabular}

Класс TeamUpdateService предназначен для обновления данных о командах через API.

\begin{xltabular}{\textwidth}{|X|X|X|X|}
	\caption{Спецификация полей класса TeamUpdateService}\label{table:TeamUpdateServiceFields}\\ \hline
	\thead{Наименование} & \thead{Область \\ видимости} & \thead{Тип данных} & \thead{Описание} \\ \hline
	\thead{1} & \thead{2} & \thead{3} & \thead{4} \\ \hline
	\endfirsthead
	\continuecaption{Продолжение таблицы \ref{table:TeamUpdateServiceFields}} \hline
	\thead{1} & \thead{2} & \thead{3} & \thead{4} \\ \hline
	\endhead
	client & private & static HttpClient & HTTP-клиент для запросов \\ \hline
	offset & private & int & Смещение для пагинации \\ \hline
	limit & private & int & Лимит для пагинации \\ \hline
	\_mapper & private & IMapper & Маппер для преобразования DTO \\ \hline
	\_dbContext & private & IApplicationDb
	Context & Контекст базы данных \\ \hline
	count & private & int & Количество команд для обработки \\ \hline
\end{xltabular}

\begin{xltabular}{\textwidth}{|X|X|X|}
	\caption{Спецификация методов класса TeamUpdateService}\label{table:TeamUpdateServiceMethods}\\ \hline
	\thead{Наименование} & \thead{Метод доступа} & \thead{Описание} \\ \hline
	\thead{1} & \thead{2} & \thead{3} \\ \hline
	\endfirsthead
	\continuecaption{Продолжение таблицы \ref{table:TeamUpdateServiceMethods}} \hline
	\thead{1} & \thead{2} & \thead{3} \\ \hline
	\endhead
	UpdateTeamsAsync & public & Асинхронное обновление данных о командах \\ \hline
	SaveToDatabase & private & Сохранение данных в базу данных \\ \hline
	UpdateTeamNameAsync & public & Асинхронное обновление имен команд \\ \hline
\end{xltabular}

Класс TournamentUpdateService предназначен для обновления данных о турнирах через API.

\begin{xltabular}{\textwidth}{|X|X|X|X|}
	\caption{Спецификация полей класса TournamentUpdateService}\label{table:TournamentUpdateServiceFields}\\ \hline
	\thead{Наименование} & \thead{Область \\ видимости} & \thead{Тип данных} & \thead{Описание} \\ \hline
	\thead{1} & \thead{2} & \thead{3} & \thead{4} \\ \hline
	\endfirsthead
	\continuecaption{Продолжение таблицы \ref{table:TournamentUpdateServiceFields}} \hline
	\thead{1} & \thead{2} & \thead{3} & \thead{4} \\ \hline
	\endhead
	client & private & static HttpClient & HTTP-клиент для запросов \\ \hline
	offset & private & int & Смещение для пагинации \\ \hline
	limit & private & int & Лимит для пагинации \\ \hline
	\_mapper & private & IMapper & Маппер для преобразования DTO \\ \hline
	\_dbContext & private & IApplicationDb
	Context & Контекст базы данных \\ \hline
	count & private & int & Количество турниров для обработки \\ \hline
\end{xltabular}

\begin{xltabular}{\textwidth}{|X|X|X|}
	\caption{Спецификация методов класса TournamentUpdateService}\label{table:TournamentUpdateServiceMethods}\\ \hline
	\thead{Наименование} & \thead{Метод доступа} & \thead{Описание} \\ \hline
	\thead{1} & \thead{2} & \thead{3} \\ \hline
	\endfirsthead
	\continuecaption{Продолжение таблицы \ref{table:TournamentUpdateServiceMethods}} \hline
	\thead{1} & \thead{2} & \thead{3} \\ \hline
	\endhead
	UpdateTournamentAsync & public & Асинхронное обновление данных о турнирах \\ \hline
	UpdateTournamentMin
	FinishedAsync & public & Асинхронное обновление данных о завершенных турнирах \\ \hline
	UpdateTournamentMin
	UpcomingAsync & public & Асинхронное обновление данных о предстоящих турнирах \\ \hline
\end{xltabular}

\subsubsection{Спецификация классов сервиса поиска}

Класс SearchController предназначен для выполнения поиска по игрокам, командам и турнирам в базе данных.

\begin{xltabular}{\textwidth}{|X|X|X|X|}
	\caption{Спецификация полей класса SearchController}\label{table:SearchControllerFields}\\ \hline
	\thead{Наименование} & \thead{Область \\ видимости} & \thead{Тип данных} & \thead{Описание} \\ \hline
	\thead{1} & \thead{2} & \thead{3} & \thead{4} \\ \hline
	\endfirsthead
	\continuecaption{Продолжение таблицы \ref{table:SearchControllerFields}} \hline
	\thead{1} & \thead{2} & \thead{3} & \thead{4} \\ \hline
	\endhead
	\_playerRepository & private & IPlayerRepository & Репозиторий для поиска игроков \\ \hline
	\_teamRepository & private & ITeamRepository & Репозиторий для поиска команд \\ \hline
	\_tournament
	Repository & private & ITournament
	Repository & Репозиторий для поиска турниров \\ \hline
\end{xltabular}

\begin{xltabular}{\textwidth}{|X|X|X|}
	\caption{Спецификация методов класса SearchController}\label{table:SearchControllerMethods}\\ \hline
	\thead{Наименование} & \thead{Метод доступа} & \thead{Описание} \\ \hline
	\thead{1} & \thead{2} & \thead{3} \\ \hline
	\endfirsthead
	\continuecaption{Продолжение таблицы \ref{table:SearchControllerMethods}} \hline
	\thead{1} & \thead{2} & \thead{3} \\ \hline
	\endhead
	Search & public & Выполняет поиск по игрокам, командам и турнирам \\ \hline
\end{xltabular}

\subsubsection{Спецификация классов сервиса прогнозирования}

Класс app предназначен для настройки и запуска веб-приложения сервиса прогнозирования.

\begin{xltabular}{\textwidth}{|X|X|X|X|}
	\caption{Спецификация полей класса app}\label{table:appFields}\\ \hline
	\thead{Наименование} & \thead{Область \\ видимости} & \thead{Тип данных} & \thead{Описание} \\ \hline
	\thead{1} & \thead{2} & \thead{3} & \thead{4} \\ \hline
	\endfirsthead
	\continuecaption{Продолжение таблицы \ref{table:appFields}} \hline
	\thead{1} & \thead{2} & \thead{3} & \thead{4} \\ \hline
	\endhead
	app & private & Flask & Экземпляр Flask приложения \\ \hline
\end{xltabular}

\begin{xltabular}{\textwidth}{|X|X|X|}
	\caption{Спецификация методов класса app}\label{table:appMethods}\\ \hline
	\thead{Наименование} & \thead{Метод доступа} & \thead{Описание} \\ \hline
	\thead{1} & \thead{2} & \thead{3} \\ \hline
	\endfirsthead
	\continuecaption{Продолжение таблицы \ref{table:appMethods}} \hline
	\thead{1} & \thead{2} & \thead{3} \\ \hline
	\endhead
	run & public & Запускает Flask приложение \\ \hline
\end{xltabular}

Класс PredictionService предназначен для предоставления функциональности прогнозирования на основе обученной модели.

\begin{xltabular}{\textwidth}{|X|X|X|X|}
	\caption{Спецификация полей класса PredictionService}\label{table:PredictionServiceFields}\\ \hline
	\thead{Наименование} & \thead{Область \\ видимости} & \thead{Тип данных} & \thead{Описание} \\ \hline
	\thead{1} & \thead{2} & \thead{3} & \thead{4} \\ \hline
	\endfirsthead
	\continuecaption{Продолжение таблицы \ref{table:PredictionServiceFields}} \hline
	\thead{1} & \thead{2} & \thead{3} & \thead{4} \\ \hline
	\endhead
	model & private & PredictorModel & Обученная модель для прогнозирования \\ \hline
\end{xltabular}

\begin{xltabular}{\textwidth}{|X|X|X|}
	\caption{Спецификация методов класса PredictionService}\label{table:PredictionServiceMethods}\\ \hline
	\thead{Наименование} & \thead{Метод доступа} & \thead{Описание} \\ \hline
	\thead{1} & \thead{2} & \thead{3} \\ \hline
	\endfirsthead
	\continuecaption{Продолжение таблицы \ref{table:PredictionServiceMethods}} \hline
	\thead{1} & \thead{2} & \thead{3} \\ \hline
	\endhead
	predict & public & Возвращает прогноз на основе входных данных \\ \hline
	load\_model & public & Загружает модель для прогнозирования \\ \hline
\end{xltabular}

Класс PredictorModel предназначен для представления модели, используемой для прогнозирования.

\begin{xltabular}{\textwidth}{|X|X|X|X|}
	\caption{Спецификация полей класса PredictorModel}\label{table:PredictorModelFields}\\ \hline
	\thead{Наименование} & \thead{Область \\ видимости} & \thead{Тип данных} & \thead{Описание} \\ \hline
	\thead{1} & \thead{2} & \thead{3} & \thead{4} \\ \hline
	\endfirsthead
	\continuecaption{Продолжение таблицы \ref{table:PredictorModelFields}} \hline
	\thead{1} & \thead{2} & \thead{3} & \thead{4} \\ \hline
	\endhead
	model & private & Any & Модель машинного обучения \\ \hline
\end{xltabular}

\begin{xltabular}{\textwidth}{|X|X|X|}
	\caption{Спецификация методов класса PredictorModel}\label{table:PredictorModelMethods}\\ \hline
	\thead{Наименование} & \thead{Метод доступа} & \thead{Описание} \\ \hline
	\thead{1} & \thead{2} & \thead{3} \\ \hline
	\endfirsthead
	\continuecaption{Продолжение таблицы \ref{table:PredictorModelMethods}} \hline
	\thead{1} & \thead{2} & \thead{3} \\ \hline
	\endhead
	predict & public & Выполняет прогноз на основе входных данных \\ \hline
	load & public & Загружает модель из файла \\ \hline
\end{xltabular}

\subsection{Настройка взаимодействия сервисов}

Взаимодействие между различными микросервисами веб-платформы для анализа и визуализации статистических данных киберспортивной игры Counter-Strike 2 осуществляется через API Gateway, реализованный с использованием Ocelot. API Gateway обеспечивает маршрутизацию запросов к соответствующим сервисам, обеспечивая централизованный входной узел для всех API-запросов.

Каждый микросервис запущен в отдельном контейнере и прослушивает свой порт:

\begin{itemize}
	\item API Gateway: 5000
	\item Сервис предоставления данных: 5001
	\item Сервис прогнозирования: 5002
	\item Сервис поиска: 5003
	\item Сервис анализа .dem файла: 5004
	\item Сервис поиска информации о турнирах: 5005
	\item База данных: 5432
	\item Frontend: 3000
	\item Redis: 6379
	\item RabbitMQ: 5672
	\item Prometheus: 9090
\end{itemize}

Конфигурация API Gateway находится в файле ocelot.json. В этом файле определены правила маршрутизации (ReRoutes) для различных сервисов. Каждый маршрут содержит информацию о схеме, хосте, портах и методах HTTP, которые должны использоваться для маршрутизации запросов.

Пример конфигурации ocelot.json для сервиса предоставления данных представлен на рисунке \ref{list:ocelot}.

\begin{figure}
	\begin{lstlisting}[language=Python]
		{
			"ReRoutes": [
			{
				"DownstreamPathTemplate": "/api/data/{everything}",
				"DownstreamScheme": "https",
				"DownstreamHostAndPorts": [
				{
					"Host": "localhost",
					"Port": 5001
				}
				],
				"UpstreamPathTemplate": "/api/data/{everything}",
				"UpstreamHttpMethod": [ "Get", "Post", "Put", "Delete" ],
				"FileCacheOptions":  { "TtlSeconds": 30 }
			},
			...
			]
		}
	\end{lstlisting}
	\caption{Пример конфигурации ocelot.json}
	\label{list:ocelot}
\end{figure}

Из данной конфигурации видно следующее:

\begin{itemize}
	\item DownstreamPathTemplate: Шаблон пути, по которому API Gateway будет маршрутизировать запросы к внутренним сервисам. В данном случае шаблон пути для downstream сервиса /api/data/{everything} указывает, что все запросы, соответствующие этому шаблону, будут перенаправляться на внутренний сервис.
	\item DownstreamScheme: Схема (http или https), которая будет использоваться для соединения с downstream сервисом. В этом случае используется https.
	\item DownstreamHostAndPorts: Список хостов и портов, на которые будут перенаправляться запросы. В данном случае запросы будут перенаправляться на локальный хост (localhost) с портом 5001.
	\item UpstreamPathTemplate: Шаблон пути, по которому API Gateway будет принимать запросы от клиентов. В данном случае шаблон пути для upstream сервиса /api/data/{everything} указывает, что все запросы, соответствующие этому шаблону, будут приняты API Gateway и перенаправлены на downstream сервис.
	\item UpstreamHttpMethod: Список методов HTTP, которые могут использоваться для запросов. В этом случае это методы GET, POST, PUT и DELETE.
	\item FileCacheOptions: Опции кэширования для файлов. В данном случае указано, что кэширование будет действовать в течение 30 секунд ("TtlSeconds": 30).
\end{itemize}

\subsection{Тестирование веб-платформы}

\subsubsection{Unit тестирование сервисов}

Unit тестирование сервисов веб-платформы является важным этапом обеспечения качества и надежности системы. Unit тесты позволяют изолировать и проверить функциональность отдельных компонентов системы, обеспечивая уверенность в их корректной работе. Ниже приведены ключевые аспекты unit тестирования для различных сервисов платформы.

\subsubsection{Описание unit тестов сервиса анализа .dem файла}

Для тестирования функциональности сервиса анализа .dem файла, были написаны unit тесты, которые проверяют корректность выполнения вычислений рейтингов игроков и команд, считывания и сохранения различных событий и прочее.

На рисунке \ref{fig:-UnitTestParser} представлены unit тесты сервиса анализа .dem файла.

\begin{figure}
	\centering
	\includegraphics[width=0.9\linewidth]{"images/UnitTestParser"}
	\caption{Unit тесты сервиса анализа .dem файла}
	\label{fig:-UnitTestParser}
\end{figure}

\subsubsection{Описание unit тестов сервиса предоставления данных}

Для тестирования функциональности сервиса предоставления данных, были написаны unit тесты, которые проверяют корректность выполнения запросов к базе данных, возврата списков игроков и команд, а также получения информации о текущих и завершенных турнирах и матчах. В тестах использовалась библиотека Moq для создания mock-объектов контекста базы данных.

На рисунке \ref{fig:-UnitTestAPI} представлены unit тесты сервиса предоставления данных.

\begin{figure}
	\centering
	\includegraphics[width=0.9\linewidth]{"images/UnitTestAPI"}
	\caption{Unit тесты сервиса предоставления данных}
	\label{fig:-UnitTestAPI}
\end{figure}

\subsubsection{Описание unit тестов сервиса поиска информации о турнирах}

Для тестирования функциональности сервиса поиска информации о турнирах были написаны unit тесты, которые проверяют корректность выполнения запросов к базе данных, обработки данных о текущих и завершённых турнирах, а также обновления информации о турнирах. В тестах использовалась библиотека Moq для создания mock-объектов контекста базы данных и маппера объектов. 

На рисунке \ref{fig:-UnitTestDataParser} представлены unit тесты сервиса поиска информации о турнирах.

\begin{figure}
	\centering
	\includegraphics[width=0.9\linewidth]{"images/UnitTestDataParser"}
	\caption{Unit тесты сервиса поиска информации о турнирах}
	\label{fig:-UnitTestDataParser}
\end{figure}

\subsubsection{Интеграционное тестирование веб-платформы}

Интеграционное тестирование веб-платформы направлено на проверку взаимодействия различных компонентов системы и их корректную работу в совокупности. В процессе тестирования проверяются интеграции между фронтендом и бэкендом, взаимодействие с внешними API, а также корректная работа базы данных и всех модулей в целом.

Для проведения интеграционного тестирования были использованы следующие подходы и инструменты:

\begin{itemize}
	\item Тестирование REST API - с использованием Postman были написаны и автоматизированы тесты для проверки всех конечных точек API, их корректных ответов и обработки ошибок.
	\item Нагрузочное тестирование - с использованием Postman были созданы сценарии нагрузочного тестирования для проверки производительности системы под высокой нагрузкой. Это позволило оценить, как система справляется с большим количеством одновременных запросов и определить её пределы производительности.
	\item Контейнеризация и оркестрация - для обеспечения изоляции и управляемости тестовых окружений использовались Docker и Kubernetes. Это позволило создать реплицируемые и изолированные тестовые окружения, обеспечивая воспроизводимость тестов.
\end{itemize}

На рисунке \ref{fig:-Postman} показан процесс нагрузочного тестирования веб-платформы.

\begin{figure}
	\centering
	\includegraphics[width=1\linewidth]{"images/Postman"}
	\caption{Нагрузочное тестирование веб-платформы}
	\label{fig:-Postman}
\end{figure}