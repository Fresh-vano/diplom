\newsection
\section*{РЕФЕРАТ}

Объем работы равен \formbytotal{page}{страниц}{е}{ам}{ам}. Работа содержит \formbytotal{figurecnt}{иллюстраци}{ю}{и}{й}, \formbytotal{tablecnt}{таблиц}{у}{ы}{}, \formbytotal{bibcnt}{библиографическ}{ий источник}{их источника}{их источников} и \formbytotal{imglistcnt}{лист}{}{а}{ов} графического материала. Количество приложений – 2. Графический материал представлен в приложении А. Фрагменты исходного кода представлены в приложении Б.

Перечень ключевых слов: киберспорт, Counter-Strike 2, визуализация данных, нейронные сети, .dem файлы, API, веб-платформа, микросервисная архитектура, контейнеризация, Docker, Kubernetes, RabbitMQ, PostgreSQL, C\#, React, Entity Framework, Golang, аналитика, статистика, REST API, frontend, backend.

Объектом разработки является веб-платформа для анализа и визуализации статистических данных киберспортивной игры Counter-Strike 2.

Целью выпускной квалификационной работы является создание эффективного инструмента для киберспортивных команд и аналитиков, который позволит улучшить игровые навыки и стратегии путем глубокого анализа игровых данных и прогнозов.

В процессе создания веб-платформы были выделены основные сущности путем анализа игровых данных, использованы классы и методы модулей, обеспечивающие работу с сущностями предметной области, а также корректную работу веб-платформы. Были разработаны следующие сервисы: API Gateway, сервис поиска информации о турнирах, сервис анализа .dem файлов, сервис предоставления данных, сервис поиска, сервис прогнозирования.

При разработке веб-платформы использовались современные технологии, такие как микросервисная архитектура, контейнеризация с Docker и Kubernetes, брокер сообщений RabbitMQ, а также базы данных PostgreSQL и ORM Entity Framework. Фронтенд часть была реализована с использованием React и MaterialUI.

Разработанная веб-платформа успешно прошла интеграционное и нагрузочное тестирование, что подтвердило её стабильную работу и высокую производительность.
\newpage
\selectlanguage{english}
\section*{ABSTRACT}
  
The volume of work is \formbytotal{page}{page}{}{s}{s}. The work contains \formbytotal{figurecnt}{illustration}{}{s}{s}, \formbytotal{tablecnt}{table}{}{s}{s}, \formbytotal{bibcnt}{bibliographic source}{}{s}{s} and \formbytotal{imglistcnt}{sheet}{}{s}{s} of graphic material. The number of applications is 2. The graphic material is presented in annex A. The fragment of the source code is provided in annex B.

List of keywords: esports, Counter-Strike 2, data visualization, neural networks, .dem files, API, web platform, microservice architecture, containerization, Docker, Kubernetes, RabbitMQ, PostgreSQL, C\#, React, Entity Framework, Golang, analytics, statistics, REST API, frontend, backend.

The object of the development is a web platform for the analysis and visualization of statistical data of the esports game Counter-Strike 2.

The purpose of the final qualifying work is to create an effective tool for esports teams and analysts, which will improve gaming skills and strategies through deep analysis of game data and predictions.

In the process of creating the web platform, the main entities were identified through the analysis of game data. Classes and methods of modules were used to ensure the work with the entities of the subject area, as well as the correct operation of the web platform. The following services were developed: API Gateway, tournament information search service, .dem file analysis service, data provision service, search service, and prediction service.

During the development of the web platform, modern technologies such as microservice architecture, containerization with Docker and Kubernetes, RabbitMQ message broker, as well as PostgreSQL databases and ORM Entity Framework were used. The frontend part was implemented using React and MaterialUI.

The developed web platform successfully passed integration and load testing, confirming its stable operation and high performance.
\selectlanguage{russian}
