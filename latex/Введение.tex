\newsection
\centertocsection{ВВЕДЕНИЕ}

Киберспорт, или профессиональный игровой спорт, за последние годы стал одним из самых быстрорастущих видов спорта в мире. Ежегодно миллионы зрителей следят за международными турнирами по таким играм, как Counter-Strike 2 (CS2), Dota 2, Warcraft III, где сотни команд соревнуются за престижные трофеи и крупные призовые фонды. В этом контексте аналитика и визуализация игровых данных становятся неотъемлемой частью подготовки к матчам и стратегического планирования.

Counter-Strike 2 — это одна из ведущих киберспортивных дисциплин, известная своей глубокой тактической составляющей и требующая от игроков высокого уровня индивидуального мастерства и командной работы. Анализ игровых данных, таких как статистика игроков, результаты матчей и метрики производительности, играет ключевую роль в разработке стратегий и принятии решений. Использование данных позволяет командам и тренерам выявлять слабые и сильные стороны, оптимизировать тактики и прогнозировать результаты будущих матчей.

\emph{Цель настоящей работы} – разработка веб-платформы для анализа и визуализации статистических данных киберспортивной игры Counter-Strike 2. Данная платформа должна предоставить киберспортивным командам, тренерам и аналитикам мощный инструмент для глубокого анализа игровых метрик и поддержки принятия решений на основе данных. Для достижения поставленной цели необходимо решить \emph{следующие задачи:}
\begin{itemize}
\item провести анализ предметной области;
\item разработать концептуальную модель веб-платформы;
\item создать базу данных всех киберспортивных матчей;
\item создать нейронную сеть для прогнозирования матчей;
\item реализовать вывода статистики на веб-платформу;
\item реализовать анализа метрик из .dem записей игр.
\end{itemize}

\emph{Структура и объем работы.} Отчет состоит из введения, 4 разделов основной части, заключения, списка использованных источников, 2 приложений. Текст выпускной квалификационной работы равен \formbytotal{page}{страниц}{е}{ам}{ам}.

\emph{Во введении} сформулирована цель работы, поставлены задачи разработки, описана структура работы, приведено краткое содержание каждого из разделов.

\emph{В первом разделе} на стадии описания предметной области собирается информация о сфере, для которой создается веб-платформа, и о существующих решениях в этой области.

\emph{Во втором разделе} на стадии технического задания приводятся требования к разрабатываемой веб-платформе.

\emph{В третьем разделе} на стадии технического проектирования представлены проектные решения для веб-платформы.

\emph{В четвертом разделе} приводится список классов и их методов, использованных при разработке сайта, производится тестирование разработанного сайта.

В заключении излагаются основные результаты работы, полученные в ходе разработки.

В приложении А представлен графический материал.

В приложении Б представлены фрагменты исходного кода. 
