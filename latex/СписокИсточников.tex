\newsection
\centertocsection{СПИСОК ИСПОЛЬЗОВАННЫХ ИСТОЧНИКОВ}

%\begin{hyphenrules}{nohyphenation} %отключение переноса слов в содержании
\begin{thebibliography}{9}
	\bibitem{cybersport} Панкина В. В., Хадиева Р. Т. Киберспорт как феномен XXI века~/ Физическая культура. Спорт. Туризм. Двигательная рекреация.~– Челябинск~: 2016.~– С. 34-38.~– Текст~: непосредственный.
	\bibitem{cybersport1} Демчук, Д. А. Развитие киберспорта в современной России~/ Исследования молодых ученых : материалы XLVII Междунар. науч. конф.~– Казань~: 2022.~– С. 37-40.~– Текст~: непосредственный.
	\bibitem{cybersport2} Гончаренко Д. И., Бровкин А. П. Сущность и содержание понятий «киберспорт» и «компьютерный спорт»~/ Педагогико-психологические и медико-биологические проблемы физической культуры и спорта~– Москва~: 2022.~– С. 84-91.~– Текст~: непосредственный.
	\bibitem{csgo} Counter-Strike: Global Offensive : Valve Developer Community : сайт. - URL: https://developer.valvesoftware.com/wiki/Category:Counter-Strike:\_Global\_Offensive (дата обращения: 28.03.2024).~– Текст~: электронный.
	\bibitem{hltv} HLTV : About us : сайт. - URL: https://www.hltv.org/about (дата обращения: 29.03.2024).~– Текст~: электронный.
	\bibitem{liquipedia} Counter-Strike Wiki : Liquipedia : сайт. - URL: https://liquipedia.net/counterstrike/Main\_Page (дата обращения: 29.03.2024).~– Текст~: электронный.
	\bibitem{uml} Рамбо Д., Якобсон И., Буч Г. Введение в UML от создателей языка~/ Д. Рамбо, И. Якобсон, Г. Буч.~– Москва~: ДМК-Пресс, 2015.~– 496 с.~– ISBN 978-5-94074-644-7.~– Текст~: непосредственный.
	\bibitem{uml1} Гома Х. UML Проектирование систем реального времени, распределенных и параллельных приложений~/ Х. Гома.~– Москва~: ДМК-Пресс, 2016.~– 700 с.~– ISBN 978-5-97060-220-1.~– Текст~: непосредственный.
	\bibitem{architecture} Клеппман, М. Высоконагруженные приложения. Программирование, масштабирование, поддержка~/ М. Клеппман.~– Санкт-Петербург~: Питер, 2018.~– 640 с.~– ISBN 978-5-44-610512-0.~– Текст~: непосредственный.
	\bibitem{gost1} ГОСТ 19.102-77 : ГОСТ 19.102-77 ЕСПД. Стадии разработки : сайт. - URL: https://cosced.ru/wp-content/uploads/2020/10/ГОСТ-19.102-77-Стадии-разработки.pdf (дата обращения: 23.03.2024).~– Текст~: электронный.
	\bibitem{gost2} ГОСТ 34.601-90 : ГОСТ 34.601-90 Информационная технология (ИТ). - URL: https://www.astoni.ru/upload/iblock/2d4/GOST-34.601\_90.pdf (дата обращения: 23.03.2024).~– Текст~: электронный.
	\bibitem{architecture1} Баланов, А. Построение микросервисной архитектуры и разработка высоконагруженных приложений. Учебное пособие~/ А. Баланов.~– Москва~: Лань, 2024.~– 244 с.~– ISBN 978-5-507-48747-9.~– Текст~: непосредственный.
	\bibitem{clearCode}	Мартин, Р. Чистый код. Создание, анализ и рефакторинг~/ Р. Мартин.~– Санкт-Петербург~: Питер, 2020.~– 464 с.~– ISBN 978-5-4461-0960-9.~– Текст~: непосредственный.
	\bibitem{clearCode1}	Webber, J. REST in Practice. Hypermedia and Systems Architecture~/ J. Webber.~– Санкт-Петербург~: Питер, 2010.~– 448 с.~– ISBN 978-0-5968-0582-1.~– Текст~: непосредственный.
	\bibitem{gateway}	Magnus L. Microservices with Spring Boot and Spring Cloud - Second Edition. Build resilient and scalable microservices using Spring Cloud, Istio, and Kubernetes~/ L. Magnus.~– Бирмингем~: Packt Publishing, 2024.~– 774 с.~– ISBN 978-1-8010-7297-7.~– Текст~: непосредственный.
	\bibitem{design}	Сюй А. System Design. Подготовка к сложному интервью~/ А. Сюй.~– Санкт-Петербург~: Питер, 2022.~– 304 с.~– ISBN 978-5-4461-1816-8.~– Текст~: непосредственный.
	\bibitem{rest} Перальта Х., Антонио А. Микросервисы и API~/ Х. Перальта., А. Антонио~– Санкт-Петербург~: Питер, 2017.~– 703 с.~– ISBN 978-5-4461-2094-9.~– Текст~: непосредственный.
	\bibitem{rest1} Любанович Б. FastAPI: веб-разработка на Python~/ Б. Любанович.~– Москва~: Sprint Book, 2024.~– 288 с.~– ISBN 978-601-08-3847-5.~– Текст~: непосредственный.
	\bibitem{https} Хоффман Э. Безопасность веб-приложений~/ Э. Хоффман.~– Санкт-Петербург~: Питер, 2022.~– 527 с.~– ISBN 978-5-4461-1786-4.~– Текст~: непосредственный.
	\bibitem{redis} Nelson J. Mastering Redis~/ J. Nelson.~– Бирмингем~: Packt Publishing, 2024.~– 366 с.~– ISBN 978-1-78398-818-1.~– Текст~: непосредственный.
	\bibitem{redis1} Palmer M. Instant Redis Persistence~/ M. Palmer.~– Бирмингем~: Packt Publishing, 2024.~– 50 с.~– ISBN 978-1-78328-021-6.~– Текст~: непосредственный.
	\bibitem{rabbitmq} RabbitMQ : RabbitMQ  documentation : сайт. - URL: https://www.rabbitmq.com/docs (дата обращения: 17.04.2024).~– Текст~: электронный.
	\bibitem{PostgreSQL} Ганс-Юрген Ш. PostgreSQL 11. Мастерство разработки~/ Ш. Ганс-Юрген.~– Москва~: ДМК-Пресс, 2019.~– 352 с.~– ISBN 978-5-97060-671-1.~– Текст~: непосредственный.
	\bibitem{PostgreSQL1} Домбровская Г., Бейликова А., Новиков Б. Оптимизация запросов PostgreSQL~/ Г. Домбровская, А. Бейликова, Б. Новиков~– Москва~: ДМК-Пресс, 2022.~– 278 с.~– ISBN 978-5-97060-963-7.~– Текст~: непосредственный.
	\bibitem{prometheus} Пивотто Ж., Бразил Б. Запускаем PROMETHEUS. Мониторинг инфраструктуры и приложений~/ Ж. Пивотто, Б. Бразил~– Москва~: ДМК-Пресс, 2023.~– 392 с.~– ISBN 978-6-01810-341-4.~– Текст~: непосредственный.
	\bibitem{docker} Моуэт Э. Использование Docker~/ Э. Моуэт.~– Москва~: ДМК-Пресс, 2017.~– 354 с.~– ISBN 978-5-97060-426-7.~– Текст~: непосредственный.
    \bibitem{kubernetes}	Onur Y., Sathsara S. Serverless Architectures with Kubernetes~/ Y. Onur., S. Sathsara~– Бирмингем~: Packt Publishing, 2024.~– 474 с.~– ISBN 978-1-8389-8327-7.~– Текст~: непосредственный.
	\bibitem{Ocelot} Ocelot : Ocelot 23.2 : сайт. - URL: https://ocelot.readthedocs.io/en/latest/index.html (дата обращения: 07.04.2024).~– Текст~: электронный.
	\bibitem{automapper} AutoMapper : AutoMapper documentation : сайт. - URL: https://docs.automapper.org/en/stable/ (дата обращения: 14.04.2024).~– Текст~: электронный.
	\bibitem{entityFramework} Entity Framework : Центр документации Entity Framework : сайт. - URL: https://learn.microsoft.com/ru-ru/ef/ (дата обращения: 14.04.2024).~– Текст~: электронный.
	\bibitem{jsonNET} Json.NET : Popular high-performance JSON framework for .NET : сайт. - URL: https://www.newtonsoft.com/json (дата обращения: 20.04.2024).~– Текст~: электронный.
	\bibitem{goland} Goland : Effective Go : сайт. - URL: https://go.dev/doc/effective\_go (дата обращения: 22.04.2024).~– Текст~: электронный.
	\bibitem{parallel}	Террелл, Р. Конкурентность и параллелизм на платформе .NET. Паттерны эффективного проектирования~/ Р. Террелл.~– Санкт-Петербург~: Питер, 2019.~– 624 с.~– ISBN 978-5-4461-1072-8.~– Текст~: непосредственный.
	\bibitem{algorinms} Бхаргава, А. Грокаем алгоритмы. Иллюстрированное пособие для программистов и любопытствующих~/ А. Бхаргава.~– Санкт-Петербург~: Питер, 2022~– 288 с.~– ISBN 978-5-4461-0923-4.~– Текст~: непосредственный.
	\bibitem{demoinfocs} demoinfocs-golang : demoinfocs-golang - Counter-Strike 2 \& CS:GO Demo Parser : сайт. - URL: https://github.com/markus-wa/demoinfocs-golang (дата обращения: 23.04.2024).~– Текст~: электронный.
	\bibitem{python} Python : Python 3.12.3 documentation : сайт. - URL: https://docs.python.org/3/index.html (дата обращения: 25.04.2024).~– Текст~: электронный.
	\bibitem{python1} Лусиану Р. Python. К вершинам мастерства. Лаконичное и эффективное программирование~/ Р. Лусиану~– Москва~: ДМК-Пресс, 2022.~– 898 с.~– ISBN 978-5-97060-885-2.~– Текст~: непосредственный.
	\bibitem{prometheus1} prometheus-net : .NET library to instrument your code with Prometheus metrics : сайт. - URL: https://github.com/prometheus-net/prometheus-net (дата обращения: 28.04.2024).~– Текст~: электронный.
	\bibitem{react} Порселло, Б. React. Современные шаблоны для разработки приложений~/ Б. Порселло.~– Санкт-Петербург~: Питер, 2022.~– 320 с.~– ISBN 978-5-4461-1492-4.~– Текст~: непосредственный.
	\bibitem{javaScript} Скотт, А. Разработка на JavaScript. Построение кроссплатформенных приложений с помощью GraphQL, React~/ А. Скоттарланд.~– Санкт-Петербург~: Питер, 2021.~– 320 с.~– ISBN 978-5-4461-1462-7.~– Текст~: непосредственный.
	\bibitem{YandexCloud} Yandex Cloud : Yandex Cloud — надежное облако для вашего бизнеса : сайт. - URL: https://yandex.cloud/ru/ (дата обращения: 30.04.2024).~– Текст~: электронный.
	\bibitem{kubernetes1} Kubernetes : Kubernetes Documentation : сайт. - URL: https://kubernetes.io/docs/home/ (дата обращения: 03.05.2024).~– Текст~: электронный.
	\bibitem{rabbitmq1} RabbitMQ : RabbitMQ 3.13 Documentation : сайт. - URL: https://www.rabbitmq.com/docs (дата обращения: 04.05.2024).~– Текст~: электронный.
	\bibitem{materisalUI} Material UI : Material UI components : сайт. - URL: https://mui.com/material-ui/all-components/ (дата обращения: 20.03.2024).~– Текст~: электронный.
\end{thebibliography}
%\end{hyphenrules}
