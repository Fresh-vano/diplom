\newsection
\section*{ОБОЗНАЧЕНИЯ И СОКРАЩЕНИЯ}

БД -- база данных.

ИС -- информационная система.

ИТ -- информационные технологии.

ПО -- программное обеспечение.

РП -- рабочий проект.

СУБД -- система управления базами данных.

ТЗ -- техническое задание.

ТП -- технический проект.

REST API -- интерфейс программирования приложений, основанный на принципах REST (Representational State Transfer), который используется для взаимодействия между различными системами через HTTP.

CS2 (Counter-Strike 2) -- многопользовательская онлайн-игра, разработанная компанией Valve.

HTTP -- протокол передачи гипертекста (HyperText Transfer Protocol), используемый для передачи данных в интернете.

DTO (Data Transfer Object) -- объект передачи данных, используемый для передачи данных между различными слоями приложения.

MVP (Most Valuable Player) -- наиболее ценный игрок, термин, используемый для обозначения лучшего игрока в матче или турнире.

UML (Unified Modelling Language) -- язык графического описания для объектного моделирования в области разработки программного обеспечения.

Frontend -- клиентская часть приложения, которая отвечает за взаимодействие с пользователем и отображение данных.

Backend -- серверная часть приложения, которая отвечает за обработку данных, взаимодействие с базой данных и реализацию бизнес-логики.

ORM (Object-Relational Mapping) -- технология программирования, которая позволяет преобразовывать данные между несовместимыми типами систем, используя объектно-ориентированные языки программирования. ORM упрощает взаимодействие между объектами в приложении и записями в реляционной базе данных, обеспечивая разработчикам возможность работать с базой данных, как если бы она была коллекцией объектов.