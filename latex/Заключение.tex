\newsection
\centertocsection{ЗАКЛЮЧЕНИЕ}

В данной выпускной квалификационной работе был разработан и внедрен комплексный инструмент для анализа и визуализации статистических данных киберспортивной игры Counter-Strike 2 (CS2). Данный проект направлен на улучшение игровых навыков и стратегий киберспортивных команд путем предоставления глубокого анализа игровых данных и прогнозов.

Основные результаты работы:

\begin{enumerate}
	\item Проведен анализ предметной области киберспорта и выявлены ключевые метрики, необходимые для анализа и прогнозирования игровых результатов.
	\item Разработана и реализована база данных для хранения информации о матчах, игроках, командах и турнирной статистике.
	\item Создана нейронная сеть для прогнозирования исходов матчей на основе исторических данных и текущих игровых метрик.
	\item Реализованы удобные интерфейсы для отображения статистики и прогнозов на веб-платформе, обеспечивая интуитивно понятный доступ к информации.
	\item Осуществлен анализ метрик из .dem файлов игр для получения детальных данных о каждом матче.
	\item Интегрированы внешние API для автоматического обновления данных о турнирах, матчах и игроках, что обеспечивает актуальность и точность информации.
	\item Использованы технологии контейнеризации и оркестрации с Docker и Kubernetes для обеспечения высокой производительности и масштабируемости системы.
	\item Внедрен брокер сообщений RabbitMQ для управления потоками данных и коммуникацией между микросервисами.
	\item Проведены интеграционные и нагрузочные тестирования, которые подтвердили стабильную работу платформы под высокой нагрузкой.
\end{enumerate}

Все требования, указанные в техническом задании, были полностью реализованы. Поставленные задачи в начале разработки проекта были успешно решены, что позволило создать инновационный инструмент для киберспортивных команд и аналитиков.
	
Готовая веб-платформа предоставляет возможность детально анализировать игровые метрики, оптимизировать тактики и прогнозировать результаты будущих матчей. Это способствует улучшению игровых навыков команд и их конкурентоспособности на международной арене.