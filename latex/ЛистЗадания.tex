\newsection
\begin{center}
\large\textbf{Минобрнауки России}

\large\textbf{Юго-Западный государственный университет}
\vskip 1em
\normalsize{Кафедра программной инженерии}
\vskip 1em

\begin{flushright}
\begin{tabular}{p{.4\textwidth}}
\centrow УТВЕРЖДАЮ: \\
\centrow Заведующий кафедрой \\
\hrulefill \\
\setarstrut{\footnotesize}
\centrow\footnotesize{(подпись, инициалы, фамилия)}\\
\restorearstrut
«\underline{\hspace{1cm}}»
\underline{\hspace{3cm}}
20\underline{\hspace{1cm}} г.\\
\end{tabular}
\end{flushright}
\end{center}
\section*{ЗАДАНИЕ НА ВЫПУСКНУЮ КВАЛИФИКАЦИОННУЮ РАБОТУ
  ПО ПРОГРАММЕ БАКАЛАВРИАТА}
{\parindent0pt
  Студента \АвторРод, шифр\ \Шифр, группа \Группа
  
1. Тема «\Тема\ \ТемаВтораяСтрока» утверждена приказом ректора ЮЗГУ от \ДатаПриказа\ № \НомерПриказа.

2. Срок предоставления работы к защите \СрокПредоставления

3. Исходные данные для создания программной системы:

3.1. Перечень решаемых задач:}

\begin{enumerate}[label=\arabic*)]
\item создание базы данных всех киберспортивных матчей;
\item создание нейронной сети для прогнозирования матчей;
\item реализация вывода статистики на веб-платформу;
\item реализация анализа метрик из .dem записей игр.
\end{enumerate}

{\parindent0pt
  3.2. Входные данные и требуемые результаты для программы:}

\begin{enumerate}[label=\arabic*)]
\item Входными данными для программной системы являются: записи прошедших матчей в формате .dem; данные с информацией о турнире; метрики игроков из матчей; данные отправленные пользователем через API.
\item Выходными данными для программной системы являются: статистика игроков из матча; прогноз на выбранный матч; данные выбранного турнира; данные выбранного матча.
\end{enumerate}

{\parindent0pt

  4. Содержание работы (по разделам):
  
  4.1. Введение
  
  4.2. Анализ предметной области
  
4.3. Техническое задание: основание для разработки, назначение разработки, требования к веб-платформе, нефункциональные требования к программной системе.

4.4. Технический проект: общие сведения о программной системе, проектирование архитектуры программной системы, обоснование выбора технологий проектирования и программных средств, проектирование пользовательского интерфейса программной системы.

4.5. Рабочий проект: описание сущностей веб-платформы, настройка взаимодействия сервисов, тестирование веб-платформы.

4.6. Заключение

4.7. Список использованных источников

5. Перечень графического материала:

\begin{enumerate}[label=Лист \arabic*.]
\item Сведения о ВКРБ
\item Цель и задачи разработки
\item Концептуальная модель программной системы
\item Диаграмма прецедентов
\item Диаграмма архитектуры программной системы
\item Заключение
\end{enumerate}

\vskip 2em
\begin{tabular}{@{}p{6.8cm}C{3.8cm}C{4.8cm}}
Руководитель ВКР & \lhrulefill{\fill} & \fillcenter\Руководитель\\
\setarstrut{\footnotesize}
& \footnotesize{(подпись, дата)} & \footnotesize{(инициалы, фамилия)}\\
\restorearstrut
Задание принял к исполнению & \lhrulefill{\fill} & \fillcenter\Автор\\
\setarstrut{\footnotesize}
& \footnotesize{(подпись, дата)} & \footnotesize{(инициалы, фамилия)}\\
\restorearstrut
\end{tabular}
}
